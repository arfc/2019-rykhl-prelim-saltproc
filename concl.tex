\chapter[Future work and Proposed simulations]{Future work and Proposed 
simulations}

\section{Summary}
This work need has been shown by a summary of the current state of the 
art of \glspl{MSR} depletion simulator capabilities. The literature review in 
Chapter 1 concluded that most \gls{MSR} depletion simulators typically assume 
ideal (rather than realistically constrained) poison removal rates for the 
nuclear system performance modeling. Moreover, most of the simulators assumed 
constant extraction efficiency vector, which must be determined by the user in 
the input file and cannot be a function of other parameters. The Python 
toolkit, SaltProc v1.0, will directly couples with the SERPENT 2 Monte Carlo 
depletion code for liquid-fueled \gls{MSR} depletion simulation to enable 
realistic online reprocessing system modeling. The SaltProc v1.0 indented to 
be a universal tool for fuel composition evolution analysis in \gls{MSR} with 
taking into account the complex fuel salt reprocessing system. Such 
reprocessing system may consist of multiple components with variable removal 
efficiency and rate. Moreover, these components can be connected in series, 
parallel, or series-parallel, which will be accurately treated in the SaltProc 
v1.0. Section~\ref{sec:reproc-plant} details the generic design of the fuel 
salt reprocessing system. Section~\ref{sec:tool_design} describes SaltProc 
v1.0 architecture and design that is demanded to successfully model 
comprehensive liquid-fueled \gls{MSR} with online fuel reprocessing system. 

Figure~\ref{fig:workflow} shows an outline of this work. Current  
Chapter details each Stage of the proposed work.
 \begin{sidewaysfigure}[ht!] % replace 't' with 'b' to force it to 
 	\centering
 	\includegraphics[width=1.06\textwidth]{progress_chart.pdf} 
 	\caption{Workflow for the simulations proposed in this work.}
 	\label{fig:workflow}
 \end{sidewaysfigure}
 \FloatBarrier
 
 
\section{Stage 1: Basic online reprocessing demonstration}
At this Stage, \gls{MSR} online reprocessing capabilities have been reviewed 
and summarized (Chapter 1). SaltProc v0.1 was demonstrated for simplified 
burnup calculation for the \gls{MSBR} as a part of my M.Sc. thesis  
\cite{rykhlevskii_advanced_2018} and published paper  
\cite{rykhlevskii_modeling_2019}. These efforts illuminated depletion of the 
fuel salt in the \gls{MSBR} for 60 years of operation with taking into account 
the following processes:
\begin{enumerate}
	\item \glspl{FP} removal from the salt with fixed, ideal extraction 
	efficiency (the fuel reprocessing system removed 100\% of target poison).
	\item 100\% of $^{233}$Pa being removed and equal mass of $^{233}$U is fed 
	into the core (instantaneous $^{233}$Pa decay to $^{233}$U was assumed).
	\item Fresh fertile material ($^{232}$Th) feed to maintain the fuel salt 
	inventory constant.
\end{enumerate}
Additionally, the effect of removing fission product from the fuel salt has 
investigated separately for a different group of \glspl{FP} (noble gases, 
noble and seminoble metals, rare earth elements). As expected, removing  
fission products provides with significant neutronic benefit and enables a 
longer core lifetime. Section~\ref{sec:pre-results-msbr} described key 
findings after completing Stage 1.

\section{Stage 2: SaltProc v1.0 demonstration and validation for the TAP}
Simulating realistic multi-component fuel reprocessing system is important for 
calculating an accurate fuel salt composition. SaltProc v0.1 was completely 
refactored for modeling a complicated salt reprocessing system. To demonstrate 
SaltProc v1.0 capabilities, we have created a full-core \gls{TAP} 
\gls{MSR} model in Serpent 2 \cite{chaube_tap_2019} which was described in a 
detail in Section~\ref{sec:tap_model}. Moreover, the multi-component fuel 
reprocessing system of the \gls{TAP} was developed on this stage 
(Section~\ref{sec:stage2-demo}). Section~\ref{sec:stage2-demo} also presented 
preliminary results of Stage 2. Stage 2 demonstration case has following 
advantages over Stage 1:
\begin{itemize}
	\item SaltProc v0.1 (Stage 1) approximated the fuel salt reprocessing 
	system 
	as a single ``black'' box, which removes the entire mass (100\% removal 
	efficiency) of processed elements at once. In contrast, SaltProc v1.0 
	treats the fuel reprocessing system as a complex structure of components, 
	each removing a specific set of elements with specific extraction 
	efficiency. 
	\item SaltProc v1.0 inherently checks mass conservation at each depletion 
	step and dynamically calculates feed stream to maintain the fuel salt 
	inventory constant.
	\item SaltProc v1.0 tracks waste stream from each component.
\end{itemize}

The foremost future effort on this Stage is to implement the capability of 
switching between multiple Serpent geometries during simulation. For the 
\gls{TAP} concept the number of moderator rods in the core varies from 1332 at 
the startup to 6700 at the \gls{EOL}. The user will have an option to choose 
when SaltProc v1.0 should switch to next geometry: (1) after specific 
depletion  time (e.g., 18 months which is a common maintenance/refueling 
shutdown interval for \glspl{LWR}); (2) effective multiplication factor 
reaches specific value (e.g., $1.00<k_{eff} < 1.002$). Additionally, SaltProc 
v1.0 will correct the total fuel salt inventory in a primary loop to 
compensate for the core geometry change. That is, after each moderator rods  
reconfiguration, excess of the fuel salt will be ``drained'' in a separate 
tank, and this tank inventory will be stored in the database. Overall, the 
adjustable geometry capability will allow to realistically simulate long-term 
(60 years) operation of the \gls{TAP} reactor to obtain accurate fuel salt 
composition at different moments during operation.

Obtained at Stage 2 results will be used for code-to-code verification with  
ChemTriton/Shift results for full-core \gls{TAP} core geometry from the most 
recent \gls{ORNL} technical report TM-2017/475 \cite{betzler_assessment_2017} 
for confidence building. Notably, the fuel salt composition evolution during 
the \gls{TAP} reactor operation and corresponding core geometry are 
determinative for all next stages.

This work is developed with a test-driven development paradigm. Specifically, 
before any new functionality is implemented, a suite of tests is written, 
which as carefully define its expected behavior as possible. The code is then 
written to pass the test suite. In this way, the tool developed in this work 
is expected to be comprehensively tested in parallel with its development. 
Thus, after code-to-code verification with ChemTriton/Shift multiple-component 
integration test will be added to the test harness to make sure that future 
changes in the code will not break previous functionality.

Test problems will help comprehensively define and confirm each unit of the 
demonstration functionality. These problems will include fundamental, 
information-passing tests as well as more challenging multiple-component 
integration tests. Every unit of functionality within the toolkit will be 
tested as an integral part of development.

This milestone will result in a processing system model capable of simulating
various liquid-fueled \gls{MSR} with multi-component fuel reprocessing system 
but with constant separation efficiency, defined at runtime. Additionally, 
this stage model will demonstrate a key feature of the \gls{TAP} \gls{MSR} - 
adjusting the moderator rods configuration - which is necessary to maintain 
the reactor critical during 60 years lifetime. 

\section{Stage 3: Variable xenon extraction rate}
When Stage 2 is established, a series of extensions to the Stage 2 model will 
be pursued. These will incorporate extraction efficiencies as a function of 
many physical system design parameters (e.g., void fraction in the salt, 
helium bubble size). Mathematical correlations for the efficiencies will be 
taken from the literature \cite{peebles_removal_1968, 
gabbard_development_1974} and CFD simulations currently being conducted 
at the University of Illinois at Urbana-Champaign \cite{huff_enabling_2018}. 
For demonstration proposes, just xenon removal efficiency will be defined as a 
function of many parameters (Section~\ref{sec:gas-separ}) due to 
limited data provided in the listed literature. For other fission products  
from the \gls{TAP} reprocessing scheme (table~\ref{tab:reprocessing_list}), 
removal efficiencies will be defined based on the removal rates from the 
table, assuming time-independent extraction efficiency. This milestone will 
result in a realistic online reprocessing system model capable of modeling 
\gls{MSR} systems with parameterized, realistically achievable process rates,  
and extraction efficiencies.

Another anticipated extension will test the \gls{TAP} reactor ability to 
operate in a load-following regime. Short-term (3 days) depletion using 
SaltProc v1.0 will be performed with the core power changing in range 0-100\% 
with ramp rate 10\%/min (to be competitive with natural gas peaking plants, 
which ramp at or above 10\% of their capacity) \cite{huff_enabling_2018}. 
Figure~\ref{fig:load} shows selected load curve to demonstrate worst-case 
scenario of load-following:
\begin{enumerate}
	\item Startup with fresh fuel and operating on 100\% of \gls{HFP}
level 
	for 40 hours to reach $^{135}$Xe/$^{135}$I equilibrium;
	\item Load-following power drop (0.1 \gls{HFP}/min), from \gls{HFP} 
	to \gls{HZP};
	\item Shutdown for 8 hours\footnote{At startup. Time after shutdown when 
	$^{135}$Xe concentration would reach maximum value greatly depends on 
	neutron energy spectrum which for the \gls{TAP} concept changes 
	significantly during operation.} to reach the $^{135}$Xe peak;
	\item Load-following power rise (0.1 \gls{HFP}/min), from \gls{HZP} 
	to \gls{HFP}.
\end{enumerate}
This scenario can be considered as backing up solar power with
nuclear on a 
high-solar penetration grid (e.g., in California).
\begin{figure}[bth!] % replace 't' with 'b' to 
	\centering
	\includegraphics[width=0.8\textwidth]{load_curve.png}
	\caption{Tentative load curve for short-term load-following depletion 
	simulation for the \gls{TAP} reactor using SaltProc v1.0.}
	\label{fig:load}
\end{figure}

The depletion step time for short-term simulation will be varied in a range 
from 1 to 60 min to find a compromise between accuracy and computational cost. 
It is expected that load-following performance will be better at the \gls{BOL} 
because neutron energy spectrum thermalizes during the reactor operation. 
Thus, the short-term load-following simulation will be repeated for the 
\gls{BOL}, the middle of life, and the \gls{EOL} to assess the \gls{TAP} 
concept performance in a load-following regime during whole reactor lifetime.

Additionally, sensitivity analysis of input parameters in the xenon extraction 
correlation will be conducted to determine parameters (e.g., mass transfer 
coefficient, helium sparging rate, gas-liquid interfacial area, temperature) 
range when load-following is possible for the \gls{TAP} 
reactor in a worst-case scenario of power demand. These multiple inputs 
incorporating user-parametrized components in the fuel salt processing 
system will be collected and published in a \textit{.json}-compatible database 
for use with the SaltProc v1.0 to encourage further research in this area.

\section{Stage 4: Prototype design for the Xe removal system}
As the model becomes capable of incorporating user-parametrized components 
with correlation-based extraction efficiency for helium sparging component, 
suitable sparger design will be determined and described. These design 
parameters will be based on parameters range (i.e., helium sparging rate) 
obtained from the previous Stage. The ultimate objective of the design is to 
ensure load-following operation during most of the operation period when 
minimizing the fuel salt inventory. That is, constrained optimization problem 
have to be solved to minimize total fuel salt volume outside of the core. The 
target design parameters for the sparger including but not limited to: the 
volume of sparger, helium flow rate, salt flow rate, and geometry.

Additionally, nuclear criticality safety analysis will be performed using 
MCNP6 \cite{werner_mcnp6._2018} to make sure that the selected sparger design 
has a subcritical configuration. If the sparger geometry obtained during the 
optimization process is supercritical, fission gas removal system would 
contain multiple spargers of smaller size connected in parallel. Minimized 
total fuel salt volume and sparger size are expected to be smaller at the 
\gls{BOL} and increasing steadily when neutron energy spectrum becomes softer.

\section{Stage 5: \gls{TAP} Safety Analysis}
The objective of this Stage is to characterize neutronics limits related to 
load following. High-fidelity simulations will achieve this goal with the 
Serpent 2 Monte Carlo code. Specifically, changes in safety parameters 
(Section~\ref{sec:safety-param}) will be evaluated for two time-frames:
\begin{enumerate}
	\item Long-time-scale changes in safety parameters should not compromise 
	the \gls{TAP} \gls{MSR} safety.
	\item Load-following operation at various moments of a lifetime (e.g., at
	startup, at the middle of life, at the end of life) must not result in 
	significant changes in safety parameters.
\end{enumerate}
Section~\ref{sec:safety-param-res} shown preliminary calculations of  
temperature coefficients and reactivity control system worth for the \gls{TAP} 
at startup. Next step will be developing multi-layer geometry in Serpent with 
non-uniform axial density distribution to estimate the power axial offset. 
Afterward, safety parameters will be calculated at the \gls{BOL}, the middle 
of life, and the \gls{EOL} (fuel salt composition and moderator rods 
configuration) to capture these parameters variation in a long-time-scale. 
Validation against previous work in a collaboration between Transatomic Power 
and ORNL \cite{betzler_assessment_2017, betzler_fuel_2018} will also be 
performed for confidence building. If feasible, additional analysis for 
different xenon removal efficiency (i.e., in the range from 0 to 100\%) will 
be performed to capture the effect of $^{135}$Xe concentration in the core on 
safety.

To analyze the impact of the load-following operation on \gls{TAP} concept 
safety, safety parameters calculations will be repeated for the load-following 
transient. The load-following operation must not result in a sufficient 
increase in the temperature coefficients to obtain positive or near-positive 
net reactivity feedback. Specifically, the combination of fuel and moderator 
temperature coefficients must remain strongly negative for all times during 
operation. Additionally, the reactivity worth of control rods must be  
sufficient to shut down the reactor at any point during the transient.


\section{Conclusions}
Details of gas removal and fuel salt processing systems in liquid-fueled 
\glspl{MSR} have been conceptual rather than concrete over decades. Usually, 
researchers assume ideal rather than realistically constrained poison 
extraction efficiency for reactor performance calculations. 
This work will more realistically model an online molten salt processing 
system with the focus on a gas removal system of the prospective \gls{TAP} 
\gls{MSR}. Developed as a part of this work, a Python toolkit, SaltProc, which 
couples directly with Serpent 2 Monte Carlo burnup software, will capture the 
evolution of fuel salt composition during reactor operation with an online 
fuel processing system.

Modeling and simulation of the online reprocessing system in the \gls{MSR} 
have shown promise in past research. Our work on simulating online fuel  
reprocessing for the thorium-fueled \gls{MSBR} yielded interesting results: 
notable neutron energy spectrum shift and corresponding changes in safety 
parameters during operation. Another work also showed promise results in 
modeling a simplified fuel processing system for the \gls{TAP} \gls{MSR}. 
These simulations motivate future work in modeling advanced liquid-fueled 
\gls{MSR} plant designs.

To establish a feasible system design for molten salt fuel reprocessing, a 
more advanced model of the \gls{TAP} \gls{MSR} system with adjustable core 
geometry and realistically achievable extraction efficiencies will be 
developed. Extended SaltProc v1.0 will realistically capture the dynamics of 
fuels salt composition changes with higher accuracy. SaltProc v1.0 will also 
be employed to simulate the \gls{TAP} \gls{MSR} behavior in short-term 
transients to determine the feasibility of load following. Additionally, input 
parameters such as flow rates, bubble size, and the void fraction will be 
varied to determine the range of these parameters when the load following is 
possible for the \gls{TAP} concept.

In addition to these simulations, several extensions are suggested to 
advance our preliminary work. First, the feasible design parameters of the 
sparger, critical component of the \gls{TAP} gas removal system, will be 
optimized through sensitivity analysis of geometry and system conditions. To 
guarantee criticality safety, MCNP6 simulation will be performed to define an 
appropriate sparger geometry. The second effort will focus on safety  
parameters evolution in the \gls{TAP} reactor during lifetime (60 years),  
when moderator rods configuration discretely changes. Finally, dynamics of the 
safety parameters will be investigated for a short-term case with fixed 
moderator configuration: load following for three days with the worst-case 
scenario of the power level change. 

