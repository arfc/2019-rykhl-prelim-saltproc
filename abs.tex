In the search for new ways to generate carbon-free, reliable 
base-load power, interest in advanced nuclear energy technologies, 
particularly \glspl{MSR}, has resurged with multiple new companies 
pursuing \gls{MSR} commercialization. To further develop these \gls{MSR} 
concepts, researchers need simulation tools for analyzing liquid-fueled 
\gls{MSR} depletion and fuel processing. Current \gls{MSR} modeling efforts in 
the literature usually assume ideal (e.g., 100\% of neutron poison being 
removed) rather than realistically constrained removal efficiency. This work 
proposes to create a Python package, SaltProc, which will implement 
realistically constrained extraction efficiency of fission product based on 
physical models of fuel processing components appearing in various \gls{MSR} 
systems. To demonstrate the capabilities of SaltProc, the \gls{TAP} concept 
will be simulated to capture the evolution of fuel salt composition during 
reactor operation with a realistic, physics-driven model of an online fuel 
reprocessing system. SaltProc will also be applied to determine the 
feasibility and safety of load following for the \gls{TAP} system. This 
proposal outlines extensions to this work, including investigating safety 
parameters (temperature coefficients, shutdown margin, axial power offset) 
evolution over a 60-year reactor operation lifetime and during short-term 
load-following transients.



