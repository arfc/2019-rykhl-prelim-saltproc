\chapter[Modeling approach for realistic simulation of online reprocessing]{Modeling approach for realistic simulation of online reprocessing}

\section{Reprocessing plant general concept}
All liquid-fueled \gls{MSR} designs involve varying levels of online fuel processing. Minimally, volatile gaseous fission products (e.g. Kr, Xe) escape from the fuel salt during routine reactor operation and must be captured. Additional systems might be used to enhance removal of those elements. Most designs also call for the removal of rare earth metals from the core since these metals act as neutron poisons. Some designs suggest a more complex list of elements to process (figure~\ref{fig:periodic_tab}), including the temporary removal of protactinium from the salt or other regulation of the actinide inventory in the fuel salt \cite{ahmad_neutronics_2015}. This section 
contains general chemical processing plant and gas separation system brief 
overview. As an example of generic reprocessing facility the reprocessing plant 
of single-fluid \gls{MSBR} was selected. 
\begin{figure}[htp!] % replace 't' with 'b' to 
  \centering
  \includegraphics[width=\textwidth]{periodic_map.png}
  \caption{Processing options for \gls{MSR} fuels \cite{ahmad_neutronics_2015}.}
  \label{fig:periodic_tab}
\end{figure}

\subsection{Gas separation system}
Volatile gaseous fission products (e.g. Kr, Xe) must be removed from the fuel salt 
to avoid reactor poisoning especially during starup and power maneuvering. This is 
particularly true for $^{135}$Xe, with its very large neutron capture cross section 
(2.65$\times$10$^6$ b). Tritium, xenon, and krypton are sparged from the fuel salt by 
helium introduced in a bypass stream by a bubble generator and subsequently removed 
by a gas separator (figure~\ref{fig:gas_removal_system}). Indeed, noble gases, because of their exceptional insolubility 
in the salt, will migrate promptly to any gaseous interface available. Because 
they form ideal-dilute mixture in salt (obey Henry's law), they will migrate in 
accordance with the conventional laws of mass transfer. If tiny helium bubbles 
are circulated with the fuel salt, they will absorb xenon and krypton fission 
products. The fission-product-rich bubbles of helium may then be separated from 
the salt and discharged to the off-gas system. Xenon migration to the circulating 
bubbles is in competition with xenon migration to the porous moderator graphite. 
The graphite is especially of concern because it absorbs xenon and holds it in the 
core which leads to parasitic neutron absorption. The 0.5\% target value for 
$^{135}$Xe poison fraction can be achieved when circulating helium bubbles 
0.508mm in diameter \cite{robertson_conceptual_1971}. This is accomplished by 
bypassing 10\% of the fuel salt from the pump discharge through a bubble separator 
to remove the xenon bubbles and then back into the pump suction. The average 
residence time of a bubble in the fuel loop would be 10 full cycles. Investigation 
for \gls{MSBR} has shown that helium babble size was approximately 25\% of the 
throat width (blue circle on figure~\ref{fig:bubble_separator}) and was independent 
of the gas flow rate \cite{robertson_conceptual_1971}. Consequently, it is possible 
to regulate helium bubble size by changing throat width in the bubble generator.
\begin{figure}[htp!] % replace 't' with 'b' to 
  \centering
  \includegraphics[width=\textwidth]{gas_separation.png}
  \caption{Schematic flow diagram \gls{MSBR} off-gas system 	\cite{robertson_conceptual_1971}.}
  \label{fig:gas_removal_system}
\end{figure}
\begin{figure}[htp!] % replace 't' with 'b' to 
  \centering
  \includegraphics[width=\textwidth]{msbr_bubble_generator.png}
  \caption{Preliminary concept of \gls{MSBR} bubble generator  \cite{robertson_conceptual_1971}. Blue circle shows throat width which 
  determines bubble size.}
  \label{fig:bubble_separator}
\end{figure}

To inform physics-based simulation of volatile gases removal in  
computational code a mathematical model is required. Particularly, xenon 
extraction efficiency is needed to accurately model $^{135}$Xe removal in 
a fuel salt 
depletion simulations. The gain and loss terms for dissolved in fuel salt 
xenon-135 atomic density are listed in tables~\ref{tab:xe_gain}, \ref{tab:xe_loss}. 
The stripping 
efficiency for the pump bowl has been measured during \gls{MSRE} but \gls{ORNL} report only states range of it (from 50 to 100\%) and mentioned that the results are not 
accurate \cite{kedl_development_1967}. $^{135}$Xe burnup and 
decay rates are relatively well known and can be estimated using neutronics 
computer code. The mass transfer coefficient can be estimated from CFD-models but 
there is lack of published information to inform accurate mathematical model. 
Previous research by \gls{ORNL} \cite{kedl_development_1967, engel_xenon_1971} has 
concluded that efficiency of xenon removal ($\epsilon_{Xe}$) in gas separation 
system is a function of many parameters:
\begin{align}
\epsilon_{Xe} = F (A, d_{He}, \dot{m}_{He}, \dot{m}_{salt}, h_b, T)
\end{align}
where $A$ is interface area of injected helium bubbles, $d_{He}$ is average diameter 
of helium bubbles, $\dot{m}_{He}$ is helium gas mass flow rate, $\dot{m}_{salt}$ is 
salt mass flow rate, $h_b$ is mass transfer coefficient for transferring xenon into 
circulating helium bubbles, $T$ is the salt temperature. 
\begin{table}[ht!]
\caption{$^{135}$Xe production sources and principal rate constants involved
 (reproduced from \cite{kedl_development_1967}).}
  \centering
\begin{tabularx}{\textwidth}{b | b}
\hline \textbf{$^{135}$Xe gain mechanism}      & \textbf{Principal rate 
parameters involved}  	\\
\hline Direct from Fission   & (Fission rate) $\times$ 0.3\% (for $^{235}$U fission) \\
\hline $^{135}$I decay       & (Fission rate) $\times$ 3.6\% (for $^{235}$U fission) \\
iodine-135 generation rate is 3.6\%, 
it decays to $^{135}$Xe with $\tau_{1/2}=6.68$ h & 			                    \\		\hline $^{135}$Te decay      & (Fission rate) $\times$ 2.5\% (for $^{235}$U fission) \\
tellurium-135 generation rate is 2.5\%, 
it quickly decays to $^{135}$I with $\tau_{1/2}=19$ s & 			                    \\					
\hline 
\end{tabularx}
  		\label{tab:xe_gain}
\end{table}
\begin{table}[ht!]
\caption{$^{135}$Xe loss terms and principal rate constants involved
 (reproduced from \cite{kedl_development_1967}).}
  \centering
\begin{tabularx}{\textwidth}{b | b}
\hline \textbf{$^{135}$Xe loss mechanism}      & \textbf{Principal rate 
parameters involved}  	\\
\hline Decay of dissolved $^{135}$Xe ($\tau_{1/2}=9.1$ h)  & Decay constant							\\
\hline $^{135}$Xe burnup              &  Neutron flux		 					\\
dissolved xenon-135 burnup as it passes throught core  & 			            \\		\hline $^{135}$Xe transferred to off-gas system via xenon stripper & Stripping efficiency		\\
\hline $^{135}$Xe transferred into circulating He bubbles; this xenon will eventually be burnup, decay, or stripped via bubble separator & Mass transfer coefficient, decay constant, neutron flux, bubble removing efficiency		\\
\hline 
\end{tabularx}
  		\label{tab:xe_loss}
\end{table}

Similar removal techniques and relationship for removal efficiency would be 
appropriate 
for other noble gases (e.g. Kr) but coefficients for different gases would be 
different. To create realistic mathematical model medium-fidelity correlations with 
coefficients for various volatile gases (but xenon is in focus) are needed. It can be 
done experimentally or using CFD simulations. Current effort at University of 
 Illinois at Urbana-Champaign namely ``Enabling Load Following Capability in the 
Transatomic Power MSR" \cite{huff_enabling_2018} has a goal to determine those 
correlations using CFD simulations 
and verify it in small-size experiments. Obtained mathematical model might be used 
to inform realistic physic-based gases removal from fuel salt in proposed SaltProc 
tool.

\subsection{Fuel salt chemical processing facility} \label{sec:chemical_processing}
In the single-fluid \gls{MSBR} thorium, uranium, 
protactinium, and fission products are all mixed together in a single fluoride salt (FLiBe). Separation of thorium from lanthanide (atomic numbers 57 through 71) fission products is rather challenging because of their chemical similarities. The principal scheme of \gls{MSBR} reprocessing facility concept is shown in Figure~\ref{fig:material_flow}. The fuel salt is first temporarily stored for cooling and decay of the shortest lived fission products, then directed to the primary fluorinator, where most of the uranium is removed by fluorination to UF$_6$. After that, the salt is routed to an extraction column where mixture containing metallic bismuth, lithium and thorium as reductants are contacted with the salt. The remaining uranium and protactinium are reductively extracted to the bismuth, leaving a salt that only contains fission products dissolved in carrier salt (base composition of LiF-BeF$_2$-ThF$_4$). The salt then goes through a reduction column where UF$_6$ is reduced to UF$_4$ in the salt, refueling it and preparing it for return to the reactor. Refill BeF$_2$ and ThF$_4$ are also added and all residual bismuth is removed from the salt. After a final cleanup step and valence adjustment, the purified salt returns to the reactor \cite{carter_design_1972,sorensen_one-fluid_2006}.
\begin{figure}[htp!] % replace 't' with 'b' to 
  \centering
  \includegraphics[width=0.9\textwidth]{flowsheet.pdf}
  \caption{Simplified block diagram of chemical processing scheme for single-fluid \gls{MSBR} \cite{robertson_conceptual_1971, sorensen_one-fluid_2006}.}
  \label{fig:material_flow}
\end{figure}

The bismuth accommodating some uranium and protactinium is routed to a hydrofluorination column where the metallic solutes in the bismuth are oxidized into their fluoride forms in the presence of a decay salt. The decay salt, containing UF$_4$, PaF$_4$, and ThF$_4$ passes into a decay tank where $^{233}$Pa is decays to $^{233}$U. The uranium generated by protactinium decay is removed through fluorination to UF$_6$ and directed to the reduction column to refuel the purified fuel salt. A hydrofluorinator and a fluorinator can remove approximately \textbf{95\%} of the uranium from the stream.

The fully processed salt, on its way back to the reactor, has uranium added from the protactinium decay tank at the rate required to maintain or adjust the uranium concentration in the reactor (and, consequently, control the reactivity). This is performed by sparging the salt with UF$_6$ and hydrogen to produce UF$_4$ in the salt and HF gas \cite{robertson_conceptual_1971}.

The fuel salt stream from the protactinium isolation system contains only traces of protactinium and uranium (because it were separated in the system) but contains practically all of the rare earths. A share of this salt stream is bypassed to a reductive extraction process for removing rare earths. This method for rare-earth removal is based on differences in the extent to which the rare earths and thorium distribute between molten salt and liquid bismuth containing a reductant. Principal scheme of Rare Earths removal system is shown in Figure~\ref{fig:rare-earth-removal}. A molten salt flow which contains rare-earth fluorides is fed to the center of an extraction column. The salt flows countercurrent to a liquid bismuth stream which contains thorium and lithium. In the upper part of the column, a large fraction of the rare earths is reduced and transfers to the downflowing liquid metal stream. Below the feed point rare earths concentration is increased in the salt and metal streams in order to produce a concentration hight enough for disposal \cite{briggs_molten-salt_1969}.

Molten salt leaving the top of the column contains low concentration of rare earths. Fraction of this flow is returned back to the reactor while the rest is sent to an electrolytic cell complex. The net effect of the complex is to push thorium and lithium into bismuth for use as extractant and to return extracted rare earths, entering the complex with bismuth from the bottom of the cascade, to the cascade as reflux, oxidizing them out of the metal and transferring them to the returning salt stream.

Electrolytic oxidizer-reducer is required for reducing lithium and thorium fluorides into a bismuth cathode to prepare the metal streams fed to the extraction columns and for oxidizing extracted components from the metal streams leaving the columns \cite{mcneese_engineering_1971}. The electrolytic-cell complex consists of an electrolytic
cell with contactors above and below it. Both the anode and the cathode of the cell are pools of flowing bismuth, and the electrolyte is salt that does not contain large amounts of rare earths or thorium. The cathode feeds lithium to the bismuth stream that flows into the
lower contactor and extracts essentially all ($\approx$ 99\%) of the thorium out of the feed stream. The anode puts BiF$_3$ into the salt that flows into the upper contactor and oxidizes all of the rare earths out of the entering bismuth. 
\begin{figure}[htbp!]
  \centering
        \includegraphics[width=0.45\textwidth]{rare-earths-removal-system.png}
    \caption{Rare-Earth removal from a fuel salt by Reductuve Extraction \cite{briggs_molten-salt_1969}.}
    \label{fig:rare-earth-removal}
\end{figure}

\subsection{Rare Earths Separation Efficiency} \label{sec:rare_earth_eff}
Rare earth removal efficiencies were calculated during Molten Salt Reactor Experiment for a range of operating conditions to establish the importance of number of stages, separation factor, metal-to-salt flow ratio, rare earth concentration in the discard stream, location of the feed point, and the fraction of ThF$_4$ which is reduced in the electrolytic oxidizer-reducer. The thorium and lithium concentrations in the bismuth stream fed to the extraction column were both 0.0016 mole fraction. 
\begin{figure}[htbp!]
    \begin{center}
        \includegraphics[width=0.55\textwidth]{vol_flow_ratio.png}
    \end{center}
    \caption{Metal-to-Salt Volumetric Flow Ration in Opper Column (24-stage column) \cite{briggs_molten-salt_1969}.}
    \label{fig:vol-flow-ratio}
\end{figure}

The optimum feed location was at the center of the column and all results to is for this feed
location. System performance is shown in Figure~\ref{fig:vol-flow-ratio} for a range of Metal-to-Salt flow ratios and for rare-earth-thorium separation factors from 1.2 to 6. The fraction of the ThF$_4$ which is reduced in the electrolytic cell was 99\%. Operation with a bismuth flow rate of 15 gpm will result in a rare earth removal efficiency of 69\% (44-day removal time) for a separation factor of 2 and a removal efficiency of 31\% (95-day removal time) for a separation factor of 1.5.

The effect of fraction of ThF$_4$ reduced in the electrolytic cell is shown in Figure~\ref{fig:th-reduction}. A significant decrease in removal efficiency results from the fraction of ThF$_4$ reduced being decreased from 99 to 90\%, and the system becomes ineffective if less than 50\% of the ThF$_4$ is reduced.  The separation factors are 1.3 for Eu, 1.7 for Pm, 1.8 for La, 2.0 for Sm, 3.0 for Nd, and 3.5 for Ce. The removal times range from about 225 days for Eu to 30 days for Nd and Ce (Table~\ref{tab:removal_time}). 
\begin{figure}[htbp!]
  \centering
        \includegraphics[width=0.55\textwidth]{th_reduction.png}
    \caption{Rare-Earth Removal System Performance as function of Bismuth Phase Flow Rate and the Thorium Removal in the Electrolytic Cell Complex \cite{briggs_molten-salt_1969}.}
    \label{fig:th-reduction}
\end{figure}
\begin{table}[ht!]
\caption{Removal cycle times for various rare earths \cite{briggs_molten-salt_1969}.}
  \centering
\begin{tabular}{c c c}
\hline Rare  & Separation & Removal Time                        \\
       Earth & Factor     & (days)								\\
\hline Pm          & 1.7				& 63.8 					\\
\hline Nd		   & 3.0                & 30.6 					\\
\hline Sm		   & 2.0                & 43.5 					\\
\hline La		   & 1.8                & 47.7 					\\
\hline Eu		   & 1.3                & 222.0 				\\
\hline Ce		   & 3.5                & 30.3 					\\
\hline 
\end{tabular}
  		\label{tab:removal_time}
\end{table}

The rare earth, which determine the operating conditions for the removal system is promethium (which has a separation factor of 1.7), since it is the strongest neutron absorber. Lanthanum is less important, and the resulting removal time ($\approx$100 days) is satisfactory.
The removal times for rare earths having separation factors greater than 2, such 
as neodymium and cerium, will be shorter than required.

Overall, based on experimental data from \gls{MSRE} chemistry-based mathematical 
models can be formulated for rare earths separation efficiency. The analysis above 
shown that efficiency of specific rare earth removal ($\epsilon_{RE}$) in chemical 
processing plant is a function of many parameters:
\begin{align}
\epsilon_{RE} = F (A, \dot{m}_{Bi}, \dot{m}_{salt}, N, K)
\end{align} 
where $A$ is a metal-to-salt interface area, $\dot{m}_{Bi}$ is bismuth mass 
flow rate, $\dot{m}_{salt}$ is salt mass flow rate, $N$ is the number of stages, 
$K$ is mass transfer coefficient at the salt / Bi-Li interface. 
This mathematical correlations can be used to inform realistic chemistry-based 
depletion simulations with proposed SaltProc tool.

\section{SERPENT overview}
SERPENT is a continuous-energy Monte Carlo neutronics software capable of solving the neutron transport problem by tracking individual neutrons within the problem geometry and using stochastic method to determine chain of events for each neutron \cite{leppanen_serpent_2015}. SERPENT has been under active development at the VTT Technical Research Centre of Finland since 2004, where it was initially conceived as a tool to simplify group constant generation in a high-fidelity Monte Carlo environment. SERPENT is now widely used transport code  with a growing user base. Now SERPENT used by more than 500 registered individuals in 155 organizations located in 37 countries around the world. The burnup calculation capability in SERPENT is based on built-in calculation routines, without using any external solvers. A restart feature allows performing fuel shuffling or applying any modifications in the input by dividing the calculation into several parts, which is crucial for online reprocessing simulations.

The latest version, SERPENT 2, supports advanced geometries and has advanced burnup capabilities, including online refueling capabilities which are necessary for neutronic computations of pebble-bed reactors and liquid-fueled \glspl{MSR} \cite{aufiero_extended_2013}. Unfortunately, built-in online refueling features are still under active development and unavailable to ordinary users. Furthermore, recently multi-physics simulations using SERPENT 2 were demonstrated, i.e. coupled calculations with thermal-hydraulics, \gls{CFD} and fuel performance codes \cite{leppanen_numerical_2015}. Two-way coupling to thermal-hydraulics, \gls{CFD}, and fuel performance codes operates on two levels: internal coupling to built-in solvers for fuel behavior and thermal-hydraulics, and external coupling via a universal multi-physics interface. 

SERPENT 2 can be effectively run in parallel on computer clusters and multi-core workstations. Parallelization is handled by thread-based OpenMP, which has the advantage that all processsors use shared memory space. Calculations can be divided into several nodes by distributed-memory \gls{MPI} parallelization. SERPENT 2  is an improvement upon SERPENT 1, and contains a complete redesign of memory management using hybrid OpenMP \cite{dagum_openmp_1998} + \gls{MPI} parallelization.  This hybrid parallelization is important in depletion calculations using computer clusters with multiple nodes, and allows to achieve significant speed-up in depletion calculations on computer clusters with more than 4'000 cores \cite{leppanen_serpent_2015}. 

All calculations herein were performed using SERPENT 2 version 2.1.30 on Blue Waters’ XE6 nodes. For cross section generation, JEFF-3.1.2 nuclear data library was employed \cite{oecd/nea_data_bank_jeff-3.1.2_2014}. 

\section{Proposed simulation tool design and capabilities}
First version of SaltProc Python tool for calculating \gls{MSR}'s fuel 
composition evolution with taking into account an online reprocessing system 
was developed as a part of my master's thesis \cite{rykhlevskii_advanced_2018, rykhlevskii_arfc/saltproc_2018}. The tool was designed to 
expand SERPENT 2 depletion capabilities for modeling liquid-fueled \gls{MSR} 
for continuous reprocessing. The Python package uses HDF5 
\cite{the_hdf_group_hierarchical_1997} to store 
data, and the PyNE Nuclear Engineering Toolkit \cite{scopatz_pyne_2012}
for SEPRENT output file parsing and nuclide naming. SaltProc is an open-source tool 
that uses a batch-wise approach to simulate continuous feeds and removals in 
\glspl{MSR}. 

Capabilities of developed tool 
working together with Monte Carlo code SERPENT 2 was demonstrated for full-core 
\gls{MSBR} design for simplified case with ideal removal efficiency (100\% of 
mass for target elements removed) \cite{rykhlevskii_modeling_2019}. Unfortunately, 
the code architecture and principal structure is not designed for flexible 
implementation of sophisticated online reprocessing systems including realistic 
physics/chemistry-based extraction efficiencies. Thus, complete refactoring of 
SaltProc using \gls{OOP} needed to create comprehensive 
generic tool to realistically model any \gls{MSR} reprocessing plant with 
taking into account variable extraction efficiencies, mass balance between 
the core and processing plant and reactivity control system.

\subsection{Proposed software architecture}
The Python toolkit will couple directly with SERPENT 2 input and output files, 
to allow the reprocessing system to couple to depletion calculation. Exist 
PyNE interfaces will be employed for SERPENT 2 output parse as well as 
newly developed interfaces for input and output handling. Python 3 \gls{OOP} 
standard features will be used to create flexible, user-friendly tool with 
great potential for further improvement and collaboration. 
Figure~\ref{fig:saltproc_class} shows tentative SaltProc class structure 
 which includes 4 main classes:
\begin{enumerate}
	\item \textit{Depcode}. Contains attributes and methods for 
	reading user's input file for depletion code, initial materials (e.g., 
	fuel and/or fertile salt) composition, principal parameters for 
	burnup simulation (e.g., neutron population and number of cycles for Monte 
	Carlo), running depletion code. The proposed work will only support SERPENT 2 
	Monte Carlo (only 
	one instance of \textit{Depcode} will be instantiated) but in principle the 
	toolkit can support any depletion code (i.e. OpenMC 
	\cite{romano_openmc_2015}).
\begin{figure}[ht!] % replace 't' with 'b' to \centering
  \includegraphics[width=\textwidth]{saltproc_class_diagram.png}
  \caption{Saltproc python package tentative class diagram with UML notation 
  and examples of object instances.}
  \label{fig:saltproc_class}
\end{figure}
	\item \textit{Simulation}. Runs SaltProc simulation step, 
	initialize, write HDF5 database, track time and convert 
	isotopic composition vector nuclides names from SERPENT to human 
	readable format. Also this class will allow to restart simulation after 
	fail by restoring data from HDF5 database or continue simulation for 
	additional depletion time.
	
	\item \textit{MaterialFlow}. Each \textit{MaterialFlow} object 
	will represent the material flowing between \textit{Process} objects. Object 
	of this class will contain an isotopic composition vector (PyNE Material object 
	initialized	from SERPENT output file \textbf{dep.m}), mass flow rate, 
	temperature, density, volume, void fraction. Exist PyNE Material 
	capabilities allows easily convert units of isotopic composition vector 
	(e.g., from atomic density provided by SERPENT to mass fraction or absolute 
	mass in desired units), decay 
	material (i.e. to model \gls{MSBR} protactinium decay tank), calculate 
	decay heat, activity and dose. The main idea of \textit{MaterialFlow} 
	object is to pass detailed information about the salt starting from 
	\gls{MSR} vessel outlet via various reprocessing facilities 
	(\textit{Processes}) which modifying \textit{MaterialFlow} object before 
	depleting the material in the next SERPENT burnup step.
		
	\item \textit{Process}. Each \textit{Process} object will represent a realistic 
	fuel processing step characterized by its throughput rate, volumetric 
	capacity, extraction efficiency for each target element (can be 
	function of many parameters), waste streams 
	and some other, specific for the particular process, parameters. Refueling 
	\textit{Process} just injecting feed \textit{MaterialFlow} usually directly 
	into the reactor core (e.g., adding fissile material with specific mass flow rate 
	to \textit{MaterialFlow} after performing all removals).
\end{enumerate}

The proposed class structure will provide outstanding flexibility in simulating 
various \glspl{MSR} designs. The library of various \textit{MaterialFlow} (e.g., 
fuel salt flow, fertile salt flow, refueling salt flow) and \textit{Process} 
(e.g., helium sparging facility, gas separator, lanthanide removal component) 
objects will be created in this work to allow user quickly create a model of 
desired reprocessing scheme. At runtime, the user will connect \textit{Process} 
objects in series or parallel with \textit{Flow} objects to form a comprehensive 
reprocessing system. The user also will be able to create his own 
objects with desired set of attributes and methods and contribute back to the 
code package using Github (https://github.com/arfc/saltproc).	
	
	

\subsection{Tentative flowchart}
Figure~\ref{fig:saltproc_flow} illustrates the online reprocessing simulation 
algorithm coupling SaltProc and SERPENT 2. To perform a depletion step, 
SaltProc reads a user-defined SERPENT 2 template file. This file contains input 
cards with parameters such as geometry, material, isotopic composition, neutron 
population, criticality cycles, total heating power, and boundary conditions.  
After the depletion calculation, SaltProc reads the depleted fuel composition 
file and stores the depleted composition isotopic vector in an HDF5 database 
and \textit{MaterialFlow} object (\textbf{CoreOut} in 
figure~\ref{fig:saltproc_flow}). This object contains isotopic composition 
vector, total volume of material, mass flow rate, density and any other 
parameters specified by user. For the simplest reprocessing case when all 
facilities located in-line (100\% of total material flow goes through 
chain of separation components), \textbf{CoreOut} flowing sequentially 
between \textit{Processes} and each \textit{Process} removing mass fraction 
of target elements with specified extraction efficiency. After that removed 
material mass will be compensated by fresh fuel salt to maintain the salt 
inventory in a primary loop constant. Finally, resulting isotopic composition 
from \textbf{ReprocOut} object will be stored in HDF5 database and 
dumped in new composition file for next SERPENT depletion run. 
\begin{figure}[ht!] % replace 't' with 'b' to \centering
  \includegraphics[width=1.05\textwidth]{saltproc_flowchart.pdf}
  \caption{Tentative generic flow chart for new Saltproc python package.}
  \label{fig:saltproc_flow}
\end{figure}

For more general case with multiple concurrent extraction processes separate 
\textit{MaterialFlow} object will be created for each branch with user-defined 
mass flow rate (e.g. 90\% of total mass flow rate flows via left branch and 
10\% through right branch). Total volume and isotopic composition vector 
for each \textit{MaterialFlow} object will be calculated as fraction of incoming 
\textbf{CoreOut} flow. Then each \textit{MaterialFlow} object will be passed via 
cascade of \textit{Processes} to separate selected chemical elements with 
specific efficiency. Finally, left-hand-side branch \textit{MaterialFlow} object 
will be merged with right-hand-side one and, similarly to previous case, fresh 
fuel salt feed will compensate loss of mass in separation facilities and keep 
fuel salt mass in a primary loop constant.

This flow chart will allow simulating the operation of a complex, multi-zone, 
multi-fluid \gls{MSR} and is sufficiently general to represent myriad reactor 
systems. The refactored version of SaltProc only will store and edit the 
isotopic composition of the fuel stream, which makes it a flexible tool to 
model any geometry: an infinite medium, a unit cell, a multi-zone simplified 
assembly, or a full core. This flexibiliity allows the user to perform 
simulations of varying fidelity and computational intensity. SaltProc is an 
open-source tool (but user needs installed SERPENT2 to use SaltProc), 
available on Github. It will leverage unit tests and continuous integration 
crucial for sustainable development. It will also have documentation
generated through Sphinx document generator for ease of use. In summary, the 
development approach of SaltProc focused on producing a generic, flexible and 
expandable tool to give the SERPENT 2 Monte Carlo code the ability to conduct 
advanced in-reactor fuel cycle analysis as well as simulate a myriad of 
online refueling and fuel reprocessing systems.

%\subsection{Reactivity control module}
%In addition, SaltProc is able to define time-dependent material feed and 
%removal rates to investigate their impacts. These rates need not be 
%constant in SaltProc. They can be defined as piecewise functions or set to 
%respond to conditions in the core. For instance, SaltProc might increase the 
%fissile material feeding rate if the effective multiplication factor, 
%$k_{eff}$, falls below a specific limit (e.g., 1.002).
%These capabilities allow SaltProc to analyze fuel cycle of a generic 
%liquid-fueled \gls{MSR}.

