\chapter[Online reprocessing modeling and safety analysis]{Online reprocessing 
modeling and safety analysis}
\section{Fuel salt reprocessing overview} \label{sec:reproc-plant}
Removing specific chemical elements from a molten salt is a complicated 
task that requires intelligent design (e.g., chemical separations equipment 
design, fuel salt flows to equipment). This section contains a brief overview 
of a generic \gls{MSR} fuel salt reprocessing system. Modeling such a system 
is the focus of the proposed work.

\subsection{Gas separation system} \label{sec:gas-separ}
Gaseous fission products (e.g., Kr, Xe) must be removed from the fuel salt 
to avoid reactor poisoning, especially during startup and power maneuvering. 
This is particularly true for $^{135}$Xe, with its extensive neutron capture 
cross section ($\approx10^6\dots10^7$ b in a thermal energy range). $^{135}$Xe 
is produced directly from fission in about 0.3\% of $^{235}$U fissions 
($\gamma_{_{^{135}Xe}}$), but an even larger fraction of $^{135}$Xe is 
produced by the decay of $^{135}$I and $^{135}$Te (table~\ref{tab:xe_gain}). 
$^{135}$I and $^{135}$Te yields from fission are 
$\gamma_{_{^{135}I}}\!=3.6$\% and $\gamma_{_{^{135}Te}}\!=2.5$\%, 
respectively. Thus, total $^{135}$Xe production  
from fission is about 6.4\% of fissions (of $^{235}$U), most of this is from 
$^{135}$I and $^{135}$Te decay. Noble gases (e.g. tritium, xenon, and krypton) can 
be removed from the fuel salt as follows:
\begin{enumerate}
    \item a bubble generator injects helium bubbles in the salt stream,
    \item noble gases migrate promptly to the helium bubbles because 
              of their extreme insolubility in the salt 
              \cite{robertson_conceptual_1971},
    \item and a gas separator discharges the fission-product-rich bubbles from 
              the salt to the off-gas system.
\end{enumerate}
Diagram~\ref{fig:xe_diagram} shows the key pathways for xenon 
production, accumulation, and removal in a typical \gls{MSR}.
%%%%%%%%%%%%%%%%%%%%%%%%%%%%%%%%%%%%%%%%%%%%%%%%%%%%%%%%%%%%%
\begin{table}[ht!]
	\caption{$^{135}$Xe production sources and principal rate constants 
		involved
		(reproduced from Kedl \emph{et al.} \cite{kedl_development_1967}).}
	\centering
	\begin{tabularx}{\textwidth}{b  b}
		\hline \textbf{$^{135}$Xe gain mechanism} & \textbf{Principal rate 
			parameters involved}  	\\ [5pt] \hline 
		Direct from fission & $\Sigma_f \gamma_{_{^{135}Xe}}\phi$ (for 
		$^{235}$U fission) \\
		yield $\gamma_{_{^{135}Xe}}\!\!\!=0.003$ & \\ [5pt] \hline 
		$^{135}$I decay     & $\Sigma_f \gamma_{_{^{135}I}}\phi$ (for 
		$^{235}$U fission) \\
		yield $\gamma_{_{^{135}Xe}}\!\!\!=0.036$, it decays to $^{135}$Xe with 
		$\tau_{1/2}=6.68$ h & 			                    \\	[5pt]	\hline 
		$^{135}$Te decay    & $\Sigma_f \gamma_{_{^{135}Te}}\phi$ (for 
		$^{235}$U 		fission) \\
		yield $\gamma_{_{^{135}Xe}}\!\!\!=0.025$, 
		it decays to $^{135}$I with $\tau_{1/2}=19$ s 
		& 			                    \\ [5pt]	\hline
	\end{tabularx}
	\label{tab:xe_gain}
\end{table}
%%%%%%%%%%%%%%%%%%%%%%%%%%%%%%%%%%%%%%%%%%%%%%%%%%%%%%%%%%%%%%%%%%%

Figure~\ref{fig:gas_removal_system} shows the principal design of the  
\gls{MSBR} gas separation system. Helium bubbles of a specific size are 
introduced in a salt stream via the primary pump bowl. These bubbles absorb 
noble gases before being separated from the salt by a gas separator. 
\gls{ORNL} proved that the \gls{MSBR} gas separation system can remove up to 
99.44\% of $^{135}$Xe \cite{briggs_molten-salt_1969}. Such an exceptional 
separation efficiency can be achieved by injecting 
$d=0.508$mm helium bubbles in the pump bowl, redirecting 10\% of the fuel salt flow
through a bubble separator to remove the bubbles, and then returning the flow back into the 
pump suction. Robertson \emph{et al.} reported that helium bubble size was 
approximately 25\% of the throat width (blue circle on 
figure~\ref{fig:bubble_separator}) and was independent of the gas flow rate 
\cite{robertson_conceptual_1971}. Consequently, it is possible to regulate the 
helium bubble size by changing the throat width in the bubble generator.
\begin{figure}[htp!] % replace 't' with 'b' to 
	\centering
	\includegraphics[width=\textwidth]{xe_diagram.pdf}
	\caption{Schematic of $^{135}$Xe circulation in a generic \gls{MSR}. 
	$x$ is the fraction of fuel salt flow from the pump discharge redirected to 
	the gas separation system, while $\epsilon_m$ and $\epsilon_{es} are the efficiencies of 
	migration (of $^{135}$Xe to the helium bubbles in the sparger) and separation (of gas in 
	the entrainment separator).
	efficiency in the entrainment separator. The orange color represents the 
	fuel salt in the primary loop,	the blue color represents the gas 
	separation system, and the gray color is moderator in the core. Fission 
	yields listed for $^{235}$U.}
	\label{fig:xe_diagram}
\end{figure}
\begin{figure}[htp!] % replace 't' with 'b' to 
  \centering
  \includegraphics[width=\textwidth]{gas_separation.pdf}
  \caption{Schematic flow diagram of the \gls{MSBR} gas separation system 
  (figure reproduced from Robertson \emph{et al.} 
  \cite{robertson_conceptual_1971}).}
  \label{fig:gas_removal_system}
\end{figure}
\begin{figure}[htp!] % replace 't' with 'b' to 
  \centering
  \includegraphics[width=0.77\textwidth]{msbr_bubble_generator.png}
  \caption{Preliminary concept of \gls{MSBR} bubble generator (figure 
  reproduced from Robertson \emph{et al.} \cite{robertson_conceptual_1971}). 
  The blue circle shows throat width, which determines bubble size.}
		\vspace{-0.25in}
  \label{fig:bubble_separator}
\end{figure}
\begin{table}[b]
	\caption{$^{135}$Xe loss terms and principal rate constants involved
		(reproduced from Kedl \emph{et al.} \cite{kedl_development_1967}).}
	\centering
	\begin{tabularx}{\textwidth}{b b}
		\hline \textbf{$^{135}$Xe loss mechanism}      & \textbf{Principal 
		rate 
			parameters involved}  	\\
		\hline Decay of dissolved $^{135}$Xe ($\tau_{1/2}=9.1$ h)  & Decay 
		constant	($\lambda$)		\\
		\hline $^{135}$Xe burnup              &  Neutron flux 
		($\phi$)		 					\\
		dissolved xenon-135 burnup as it passes throught core  
		& 			            \\		\hline $^{135}$Xe migrated to 
		helium bubbles & Removing efficiency 
		($\epsilon_m$)		\\
		\hline $^{135}$Xe transferred into circulating He bubbles; this xenon 
		will eventually be burnup, decay, or stripped via bubble separator & 
		Mass transfer coefficient ($h$), decay constant ($\lambda$), 
		neutron flux ($\phi$), bubble removing efficiency 
		($\epsilon_{es}$)		\\
		\hline 
	\end{tabularx}
	\label{tab:xe_loss}
\end{table}

To realistically model the gas separation system, a mathematical model 
describing noble gas extraction efficiency dynamics during reactor operation 
is required. Particularly, a model of xenon extraction efficiency as a 
function of sparger design parameters is needed to accurately model 
$^{135}$Xe removal in a fuel salt depletion simulation. The gain and loss 
terms for $^{135}$Xe dissolved in the fuel salt are listed in 
Tables~\ref{tab:xe_gain} and \ref{tab:xe_loss}. The removal efficiency for 
the xenon in the pump bowl was measured during \gls{MSRE} operation, but the  
technical report ORNL-4069 by Kedl-Houtzeel only stated its range (from 50\% 
to 100\%) and concluded, ``It is probably a complex parameter like the 
circulating-void fraction and depends on many reactor operational variables.'' 
\cite{kedl_development_1967}. $^{135}$Xe burnup and decay rates are well 
known. 

Peebles \emph{et al.} in ORNL-TM-2245 has reported xenon removal efficiency 
($\epsilon_{Xe}$) in a gas separation system as a function of many parameters 
\cite{peebles_removal_1968}:
\begin{align}\label{eq:gas_eff}
& \qquad\qquad \epsilon_{Xe} = \frac{1-e^{-\beta}}{1+\alpha}
	\intertext{where}
 	\alpha &= \frac{RTQ_{salt}}{HQ_{He}} \\
 	\beta &= \frac{K_L a A_C L (1+\alpha)}{Q_{salt}} \\
 	R &= \mbox{universal gas constant} \nonumber \\
 	T &= \mbox{salt temperature} \nonumber \\
 	Q_{salt}&= \mbox{volumetric salt flow rate} \nonumber \\
 	Q_{He}&= \mbox{volumetric helium flow rate} \nonumber \\
 	H &= \mbox{Henry's law constant for solute gas} \nonumber \\
 	a &= \mbox{gas-liquid interfacial area} \nonumber \\
 	A_C &= \mbox{contactor cross section} \nonumber \\
 	L &= \mbox{contactor length} \nonumber \\
  	K_L &= \mbox{liquid phase mass transfer coefficient.} \nonumber
\end{align}
Most of the input parameters for that correlation are obvious and easy to 
obtain from the system component design. The mass transfer coefficient for 
transferring xenon into helium bubbles ($K_L$) can be estimated from 
experiment or a CFD model but published information is currently insufficient 
to inform an accurate mathematical model. Thus, Peebles \emph{et al.} 
reported the mass transfer coefficient correlation for the \gls{MSBR} salt 
(LiF-BeF$_2$-ThF$_4$-UF$_4$) but for limited case. While it is out of the 
scope of this work to accurately estimate mass transfer coefficient, this work 
seeks to provide a tool which will allow the user to specify any mathematical 
model for a separation efficiency.

Equation~\ref{eq:gas_eff} would apply to other noble gases (e.g., Kr) but the 
mass transfer coefficients would be different. Current effort at the 
University of Illinois at Urbana-Champaign namely, ``Enabling Load Following 
Capability in the Transatomic Power \gls{MSR}" \cite{huff_enabling_2018}, has  
a goal to determine mass transfer coefficients for various gaseous fission 
products using CFD simulations and verify them in small experiments. As a 
result, the obtained mathematical model for gas removal efficiency will be 
used to inform a realistic physics-based fuel reprocessing model in the 
proposed SaltProc tool.

\subsection{Fuel chemical processing facility} \label{sec:chemical_processing}
In addition to noble gases, the fuel salt reprocessing system must extract 
other \glspl{FP}: noble/semi-noble metals and lanthanides. They generally have 
a relatively low capture cross section and thus absorb fewer neutrons than 
$^{135}$Xe, but their removal is crucial to guarantee normal operation. Some 
fraction of noble and semi-noble solid fission products plate out onto the 
internal surfaces of the primary loop equipment  
\cite{briggs_molten-salt_1964}. Meanwhile, lanthanides have relatively high 
solubility in the carrier salt and must be removed by chemical extraction. 

In a thorium-fueled \gls{MSR} designs, $^{232}$Th in the fuel salt absorbs 
thermal neutrons and produces $^{233}$Pa which then decays into the fissile 
$^{233}$U (figure~\ref{fig:th_u_reaction}. Protactinium presents a challenge, 
since it has a large absorption cross section in the thermal energy spectrum. 
Accordingly, $^{233}$Pa is continuously removed from the fuel salt into a 
protactinium decay tank to allow $^{233}$Pa to decay to $^{233}$U without 
poisoning the reactor. This feature allows the thorium-fueled \gls{MSR} to 
avoid neutron losses to protactinium, keeps fission products to a very low 
level, and increases the efficiency of $^{233}$U breeding. 
\begin{figure}[htp!] % replace 't' with 'b' to 
	\centering
	\includegraphics[width=0.6\textwidth]{th_u_cycle.png}
	\caption{Production of $^{233}$U from $^{232}$Th.}
	\label{fig:th_u_reaction}
\end{figure}

Many authors reported that a liquid-liquid reductive extraction process is the 
best option for removing protactinium and soluble fission products from 
molten fluoride salts \cite{briggs_molten-salt_1969, delpech_molten_2010, 
doligez_coupled_2014}. In that process, the protactinium or lanthanides can be 
selectively stripped from the salt into liquid bismuth due to different 
chemical potentials. Moreover, the \gls{MSRE} experience indicated that the 
extraction could be carried out rapidly and continuously  
\cite{whatley_engineering_1970}.

The principal scheme of the \gls{MSBR} reprocessing facility concept is shown 
in Figure~\ref{fig:material_flow}. The fuel salt is first temporarily stored 
for cooling and decay of the shortest-lived fission products, then it is 
directed to the primary fluorinator. There, most of the uranium is removed by 
fluorination to UF$_6$. After that, the salt is routed to an extraction column 
where it is combined with a mixture containing metallic bismuth,
lithium, and thorium reductants.
The remaining uranium and protactinium is  
reductively extracted to a bismuth solution, leaving a salt that only contains 
fission products dissolved in carrier salt (base composition 
LiF-BeF$_2$-ThF$_4$). The salt then goes through a reduction column where 
UF$_6$ is reduced to UF$_4$ preparing it for return to the reactor. BeF$_2$ 
and ThF$_4$ are also added, and all residual bismuth is removed from the salt. 
After a final cleanup step and valence adjustment, the purified salt returns 
to the reactor \cite{carter_design_1972, 
sorensen_one-fluid_2006}.

\begin{figure}[htp!] % replace 't' with 'b' to 
  \centering
  \includegraphics[width=1.05\textwidth]{flowsheet.pdf}
  \caption{Simplified block diagram of chemical processing scheme for 
  single-fluid \gls{MSBR} (reproduced from Sorensen 
  \cite{sorensen_one-fluid_2006}). \emph{RE} represents the rare 
  earth elements extracted from the salt.}
  \label{fig:material_flow}
\end{figure}

The bismuth accommodating some uranium and protactinium is routed to a 
hydrofluorination column where metallic solutes in the bismuth are oxidized 
into their fluoride forms in the presence of a decay salt\footnote{The decay 
salt contains UF$_4$, PaF$_4$, ThF$_4$ and fission products. Uranium produced 
after $^{233}$Pa decay is extracted and directed back into the reactor. Decay 
salt is the precursor for the waste salt as it was periodically discarded 
every 220 days.}. The decay salt, containing UF$_4$, PaF$_4$, and ThF$_4$, 
passes into a decay tank where $^{233}$Pa is decays to $^{233}$U. The uranium 
generated by protactinium decay is removed through fluorination to UF$_6$ and 
directed to the reduction column to refuel the purified fuel salt. A  
hydrofluorinator and a fluorinator can remove approximately 95\% of the 
uranium from the stream \cite{robertson_conceptual_1971}.

The fully processed salt, on its way back to the reactor, has uranium added 
from the protactinium decay tank at the rate required to maintain or adjust 
the uranium concentration in the reactor (and, consequently, control the 
reactivity). Adding fissile material is performed by sparging the salt with 
UF$_6$ and hydrogen to produce UF$_4$ in the salt and HF gas 
\cite{robertson_conceptual_1971}.

After these separation steps, the fuel salt stream from the protactinium 
isolation system contains only traces of protactinium and uranium but contains 
practically all of the rare earths. A fraction of this salt stream is 
redirected to a reductive extraction process for removing rare earths.  The 
principal scheme of a rare earth removal system is shown in  
Figure~\ref{fig:rare-earth-removal}. A molten salt flow which contains 
rare earth fluorides is fed to the center of an extraction column. The salt 
flows countercurrent to a liquid bismuth stream which contains thorium and 
lithium. In the upper part of the column, the rare earths are reduced and 
transferred to the downflowing liquid metal stream. Below the feed point, the 
rare earth concentration is increased in the salt and metal streams in order 
to produce a concentration high enough for disposal 
\cite{briggs_molten-salt_1969}.
\begin{figure}[htbp!]
	\centering
	\includegraphics[width=0.45\textwidth]{rare-earths-removal-system.png}
	\caption{Rare earth removal from a fuel salt by reductive extraction 
	(figure reproduced from Briggs \emph{et al.} 
	\cite{briggs_molten-salt_1969}).}
	\label{fig:rare-earth-removal}
\end{figure}

While it is out of the scope of proposed work to derive the accurate 
chemistry-based mathematical formula for \glspl{FP} separation efficiency, 
this work seeks to provide a flexible tool that will be able to simulate 
chemical processes in significant detail with regard to key system design 
parameters.

\section{Serpent overview}
Serpent is a continuous-energy Monte Carlo neutronics software capable of 
solving the neutron transport problem by tracking individual neutrons within 
the problem geometry and using stochastic method to determine chain of events 
for each neutron \cite{leppanen_serpent_2014}. Serpent has been under active 
development at the VTT Technical Research Centre of Finland since 2004, where 
it was initially conceived as a tool to simplify group constant generation in 
a high-fidelity Monte Carlo environment. Serpent is now a widely used 
transport code  with a growing user base. Now Serpent is used by more than 500 
registered individuals in 155 organizations located in 37 countries around the 
world. The burnup calculation capability in Serpent is based on built-in 
calculation routines, without using any external solvers. A restart feature 
enables fuel shuffling simulation or applying any modifications to the input 
by dividing the calculation into several parts, which is crucial for online 
reprocessing simulations.

The latest version, Serpent 2, supports advanced geometries and has advanced 
burnup capabilities, including online refueling capabilities which are 
necessary for neutronic computations of pebble-bed reactors and liquid-fueled 
\glspl{MSR} \cite{aufiero_extended_2013}. Unfortunately, built-in online 
refueling features are still under active development and unavailable to 
ordinary users. Also, multi-physics simulations using Serpent 2 have been  
demonstrated, including  calculations with thermal-hydraulics, \gls{CFD} and 
fuel performance codes \cite{leppanen_numerical_2015}. 

Serpent 2 can be effectively run in parallel on computer clusters and 
multi-core workstations. Parallelization is handled by thread-based OpenMP, 
which enables all processsors to use shared memory space. Calculations can be 
divided into several nodes by distributed-memory \gls{MPI} parallelization. 
Serpent 2  is an improvement upon Serpent 1, and contains a complete redesign 
of memory management using hybrid OpenMP \cite{dagum_openmp_1998} + \gls{MPI} 
parallelization.  This hybrid parallelization is important in depletion 
calculations using computer clusters with multiple nodes, and allows to 
achieve significant speed-up in depletion calculations on computer clusters 
with more than 4,000 cores \cite{leppanen_serpent_2014}. 

All calculations herein were performed using Serpent 2 version 2.1.31 on Blue 
Waters’ XE6 nodes. For cross section generation, the JEFF-3.1.2 nuclear data 
library was employed based on entirely open cross section data 
\cite{oecd/nea_jeff-3.1.2_2014}. 

\section{Proposed simulation tool design and capabilities} 
\label{sec:tool_design}
The first version of the SaltProc Python tool for calculating \gls{MSR} fuel 
composition evolution, taking into account an online reprocessing system 
was developed in 2018 as a part of the M.S. thesis preceding this proposal  
\cite{rykhlevskii_advanced_2018, rykhlevskii_arfc/saltproc_2018}. The tool was 
designed to expand Serpent 2 depletion capabilities for modeling liquid-fueled 
\glspl{MSR} with online fuel reprocessing system. SaltProc v0.1 uses HDF5 
\cite{the_hdf_group_hierarchical_1997} to store data and uses the PyNE Nuclear 
Engineering Toolkit \cite{scopatz_pyne_2012} for Serpent 2 output file parsing 
and nuclide naming. SaltProc v0.1 is an open-source Python package that uses a 
batch-wise approach to simulate continuous feeds and removals in \glspl{MSR}. 

SaltProc v0.1 only allows 100\% separation efficiency for either specific 
elements or groups of elements at the end of the specific ``cycle 
time''\footnote{The \gls{MSBR} program defined ``cycle time'' as the time 
required to remove 100\% of a target nuclide from a fuel salt 
\cite{robertson_conceptual_1971}.}. Capabilities of the developed tool, 
working with the Monte Carlo software Serpent 2, were demonstrated using the 
full-core MSBR design for a simplified case with ideal removal efficiency 
(100\% of mass for target elements removed) \cite{rykhlevskii_modeling_2019}. 
The SaltProc v0.1 architecture and the principal structure was not designed 
for flexible implementation of sophisticated online reprocessing systems, 
including realistic variable extraction efficiencies.  Proposed improvements, 
discussed below, will correct this.

For the proposed work, SaltProc v0.1 will be completely refactored using 
\gls{OOP} to create a comprehensive generic tool to realistically model 
complex \gls{MSR} fuel reprocessing systems while taking into account 
variable extraction efficiencies, time-dependent core geometry, and the mass 
balance between the core and the reprocessing plant.

\subsection{Proposed software architecture}
The SaltProc v1.0 Python toolkit will couple directly with Serpent 2 input 
and output files, to couple the reprocessing system to depletion calculation. 
Python 3 \gls{OOP} standard features will be used to create a flexible, 
user-friendly tool with great potential for further improvement and 
collaboration. Figure~\ref{fig:saltproc_class} shows the proposed SaltProc 
v1.0 class structure which includes 4 main classes:
\begin{figure}[ht!] % replace 't' with 'b' to \centering
	\includegraphics[width=1.03\textwidth]{saltproc_class_diagram.png}
	\vspace{-0.15in}
	\caption{SaltProc v1.0 python package class diagram in UML notation 
		with examples of object instances.}
	\label{fig:saltproc_class}
\end{figure}
\paragraph{Depcode.}\textit{Depcode} class contains attributes and methods for 
reading the user's input file for the depletion software, initial material 
(e.g., fuel and/or fertile salt) composition, principal parameters for burnup 
simulation (e.g., neutron population and number of cycles for Monte Carlo 
neutron transport), and running the depletion code.
\paragraph{Simulation.}\textit{Simulation} class runs Serpent depletion step, 
creates and writes HDF5 database, tracks time and converts isotopic 
composition vector nuclide names from Serpent to human-readable format.
\paragraph{MaterialFlow.}Each \textit{MaterialFlow} object represents the 
material flowing between \textit{Process} objects 
(figure~\ref{fig:matflow_obj}. All instances of this class 
contain an isotopic composition vector stored in PyNE Material object, mass 
flow rate, temperature, density, volume, and void fraction. Existing PyNE 
Material capabilities convert the units of the isotopic composition vector 
(e.g., from the atomic density provided by Serpent to a mass fraction or 
absolute mass in desired units) and decay the material (i.e., model the 
\gls{MSBR} protactinium decay tank). The main idea of the 
\textit{MaterialFlow} object is to pass detailed information about the salt 
starting at the \gls{MSR} vessel outlet throughout reprocessing components 
(\textit{Processes}), which modify the \textit{MaterialFlow} object before 
depleting the material in the next Serpent burnup step. 
\begin{figure}[ht!] % replace 't' with 'b' to 
	\centering
	\includegraphics[width=0.7\textwidth]{materialflow.pdf}
	\vspace{-0.1in}
	\caption{Schematic for passing material data between fuel processing 
	system components.}
	\label{fig:matflow_obj}
\end{figure}
\paragraph{Process.}Each \textit{Process} object represents a 
realistic fuel processing step characterized by its throughput rate, 
volumetric capacity, extraction efficiency for each target element (can be 
a function of many parameters), waste streams, and other process-specific 
parameters. Feed \textit{Process} injects fresh fuel salt 
\textit{MaterialFlow} directly into the reactor core (e.g., adding fissile 
material with a specific mass flow rate to \textit{MaterialFlow} after 
performing all removals).\\


The proposed class structure provides outstanding flexibility in simulating 
various \gls{MSR} fuel processing system designs. A library of various  
\textit{MaterialFlow} (e.g., fuel salt flow, fertile salt flow, refueling salt 
flow) and \textit{Process} (e.g., helium sparging facility, gas separator, 
lanthanide removal component) objects will be created to allow a user to 
quickly create a model of a desired reprocessing scheme. At runtime, the user 
will connect \textit{Process} objects in series or parallel with 
\textit{MaterialFlow} objects to form a comprehensive reprocessing system. The 
user will also be able to create custom objects with desired attributes and 
methods and contribute back to the code package using GitHub 
(https://github.com/arfc/saltproc).	

\subsection{Tentative flowchart}
Figure~\ref{fig:saltproc_flow} illustrates the proposed online reprocessing 
simulation algorithm coupling SaltProc v1.0 and Serpent. To perform a 
depletion step, SaltProc v1.0 reads a user-defined Serpent template file. 
This file contains input parameters such as geometry, material, isotopic 
composition, neutron population, criticality cycles, total heating power, and 
boundary conditions. SaltProc v1.0 fills in the template file and runs 
Serpent single-step depletion. After the depletion calculation, SaltProc v1.0 
reads the depleted fuel composition file into the \textit{MaterialFlow} object 
(\textit{core\textunderscore outlet} in figure~\ref{fig:saltproc_flow}). This 
object contains an isotopic composition vector, total volume of material, 
total mass, mass flow rate, density, temperature, void fraction, etc. For the 
simplest reprocessing case, when all fuel processing components are located 
in-line (100\% of total material flow goes through a chain of separation 
components), the \textit{core\textunderscore outlet} 
object is flowing sequentially between \textit{Processes} and each 
\textit{Process} is removing a mass fraction of target elements with specified 
extraction efficiency. Afterward, the removed material mass is compensated by 
fresh fuel salt to maintain the salt inventory in a primary loop. 
Finally, resulting isotopic composition after reprocessing is stored in 
HDF5 database and dumped in a new composition file for the next 
Serpent depletion run. SaltProc v1.0 also stores in database isotopic  
composition before reprocessing and waste stream from each fuel processing 
component. 
\begin{figure}[ht!] % replace 't' with 'b' to \centering
	\centering
	\includegraphics[width=1.05\textwidth]{saltproc_flowchart.pdf}
	\vspace{-0.15in}
	\caption{Tentative generic flow chart for SaltProc v1.0 python package.}
	\label{fig:saltproc_flow}
\end{figure}

For a more general case with multiple concurrent extraction processes, a 
separate \textit{MaterialFlow} object is created for each branch with a 
user-defined mass flow rate (e.g., 90\% of total mass flow rate flows via left 
branch and 10\% throughout a right branch). The total mass and isotopic 
composition vector for each \textit{MaterialFlow} object is calculated as a 
fraction of incoming \textit{core\textunderscore outlet} flow. Then each 
\textit{MaterialFlow} object is passed via a cascade of \textit{Processes} to 
separate selected chemical elements with specific efficiency. Finally, the 
left-hand-side branch \textit{MaterialFlow} object is merged with the 
right-hand-side and similarly to the previous case, fresh fuel salt feed 
compensate the loss of mass in separation facilities and keep fuel salt mass 
in a primary loop constant.

The class diagram (Figure~\ref{fig:saltproc_class}) allows to model the 
operation of a complex, multi-zone, multi-fluid \gls{MSR} and is sufficiently 
general to represent myriad reactor systems. The refactored version of 
SaltProc will only store and edit the isotopic composition of the fuel stream, 
which makes it a flexible tool to model any geometry: an infinite medium, a 
unit cell, a multi-zone simplified assembly, or a full core. This flexibility 
allows the user to perform simulations of varying fidelity and computational 
intensity. SaltProc v1.0 is an open-source tool (but a user needs Serpent 
2.1.31 installed to use SaltProc v1.0), available on Github. It leverage unit 
and continuous tests  crucial for sustainable development  
\cite{krekel_pytest_2004}. It will also have documentation generated through 
Sphinx, a documentation generator, for ease of use \cite{brandl_sphinx_2009}. 
In summary, the development approach of SaltProc v1.0 is focused on producing 
a generic, flexible and expandable tool to give the Serpent 2 Monte Carlo code 
the ability to conduct advanced in-reactor fuel cycle analysis as well as 
simulate many online refueling and fuel reprocessing systems.

%\subsection{Reactivity control module}
%In addition, SaltProc will be able to define time-dependent material feed and 
%removal rates to investigate their impacts. These rates need not be 
%constant in SaltProc. They can be defined as piecewise functions or set to 
%respond to conditions in the core. For instance, SaltProc might increase the 
%fissile material feeding rate if the effective multiplication factor, 
%$k_{eff}$, falls below a specific limit (e.g., 1.002).
%These capabilities allow SaltProc to analyze fuel cycle of a generic 
%liquid-fueled \gls{MSR}.

\section{Preliminary results}\label{sec:pre-results-msbr}
Developed as a part of my master thesis, the first version of the tool only 
was able to leverage ideal removals (e.g., 100\% of target isotope mass 
extracted). The capabilities of SaltProc were demonstrated for a full-core 
model of the \gls{MSBR} model \cite{rykhlevskii_full-core_2017, 
rykhlevskii_modeling_2019}.
Subsection~\ref{sec:msbr_reproc} summarized that preliminary work and results 
obtained from applying the previous version of SaltProc to the \gls{MSBR} core 
and reprocessing plant.

\subsection{MSBR online reprocessing analysis} \label{sec:msbr_reproc}
The \gls{MSBR} vessel has a diameter of 680 cm and a height of 610 cm. It 
contains a molten fluoride fuel-salt mixture that generates heat in the active 
core region and transports that heat to the primary heat exchanger by way of 
the primary salt pump. In the active core region, the fuel salt flows through 
channels in moderating and reflecting graphite blocks.  
Figure~\ref{fig:serpent_plan_view} shows the configuration of the 
\gls{MSBR} vessel, including the ``fission" (zone I) and ``breeding" 
(zone II) regions inside the vessel. The core has two radial zones bounded by 
a solid cylindrical graphite reflector and the vessel wall. The central zone, 
zone I, in which 13\% of the volume is fuel salt and 87\% graphite, is
composed of 1,320 graphite cells, 2 graphite control rods, and 2 
safety\footnote{ These rods needed for emergency shutdown only.} rods. The 
under-moderated zone, zone II, with 37\% of fuel salt, and the radial 
reflector, surrounds the zone I core region and serves to diminish neutron 
leakage. Zones I and II are surrounded radially and axially by fuel salt 
(figure~\ref{fig:serpent_zoneII}). This space for fuel is necessary for 
injection and flow of molten salt.
\begin{figure}[ht!] % replace 't' with 'b' to 
	\centering
	\includegraphics[width=\textwidth]{view_serpent.png}
	\caption{$XY$ (left) and $XZ$ (right) views of Serpent \gls{MSBR} model 
		(figure reproduced from Rykhlevskii \emph{et al.} 
		\cite{rykhlevskii_full-core_2017}).}
	\label{fig:serpent_plan_view}
\end{figure}
\begin{figure}[hb!] % replace 't' with 'b' to 
	\centering
	\includegraphics[width=0.7\textwidth]{ser_zone_II.png}
	\caption{Detailed view of \gls{MSBR} two zone model. 
		Yellow represents fuel salt, purple represents graphite, and aqua 
		represents the reactor vessel. Figure reproduced from Rykhlevskii 
		\emph{et al.} \cite{rykhlevskii_full-core_2017}.}
	\label{fig:serpent_zoneII}
\end{figure}

As mentioned in section~\ref{sec:chemical_processing}, the \gls{MSBR} design 
requires online reprocessing to remove neutron gaseous \glspl{FP} (Xe, Kr), 
noble metals (e.g., Se, Nb, Mo) and $^{233}$Pa every 20 seconds.  
Table~\ref{tab:reprocessing_list_msbr} summarizes a full list of 
nuclides and the cycle time used for modeling salt treatment and separations 
\cite{robertson_conceptual_1971}. The removal rates vary among nuclides in 
this reactor concept and dictate the necessary resolution of depletion 
calculations. If the depletion time intervals are very short, an enormous 
number of depletion steps are required to obtain the equilibrium composition. 
On the other hand, if the depletion  calculation time interval is too long, 
the impact of short-lived fission products is not captured. To compromise, a 3-
day time interval was selected for depletion calculations to correlate with 
the removal interval of $^{233}$Pa, and $^{232}$Th was continuously added to 
maintain the initial mass fraction of $^{232}$Th.
%%%%%%%%%%%%%%%%%%%%%%%%%%%%%%%%%%%%%%%%
\begin{table}[ht!]
	\caption{The cycle times for protactinium and fission 
		products removal from the \gls{MSBR} (reproduced from Robertson 
		\emph{et al.} 
		\cite{robertson_conceptual_1971}).}
	\begin{tabularx}{\textwidth}{x  s  x}
		\hline \textbf{Processing group} & \qquad\qquad\qquad 
		\textbf{Nuclides} & \textbf{Cycle time (at full power)} \\ \hline 
		Rare earths & Y, La, Ce, Pr, Nd, Pm, Sm, 
		Gd & 50 days \\ \qquad & Eu & 500 days \\ Noble metals & Se, 
		Nb, Mo, Tc, Ru, Rh, Pd, Ag, Sb, Te & 20 sec \\
		Seminoble metals & Zr, Cd, In, Sn & 200 days \\
		Gases & Kr, Xe & 20 sec \\ Volatile fluorides & Br, I & 60 days \\
		Discard & Rb, Sr, Cs, Ba & 3435 days \\ 
		%Salt discard & Th, Li, Be, F & 3435 days \\ 
		Protactinium & $^{233}$Pa & 3 days \\ Higher 
		nuclides & $^{237}$Np, $^{242}$Pu & 16 years \\  \hline
	\end{tabularx}
	\label{tab:reprocessing_list_msbr}
\end{table}
%%%%%%%%%%%%%%%%%%%%%%%%%%%%%%%%%%%%%%%%%

Figures~\ref{fig:keff_msbr} and \ref{fig:keff_zoomed_msbr} show the effective 
multiplication factors obtained using SaltProc v0.1 and Serpent. The effective 
multiplication factors were calculated after removing fission products listed 
in Table~\ref{tab:reprocessing_list_msbr} and adding the fertile material at 
the end of cycle time (3 days for this work). The effective multiplication 
factor fluctuates significantly as a result of the batch-wise nature of this 
online reprocessing strategy. 
\begin{figure}[ht!] 
	\centering
	\includegraphics[width=\textwidth]{keff_msbr.png}
	\caption{Effective multiplication factor dynamics for the full-core 
	\gls{MSBR} 
		model over a 60-year reactor operation lifetime (reproduced from  
		Rykhlevskii \emph{et al.} \cite{rykhlevskii_modeling_2019}).}
	\label{fig:keff_msbr}
\end{figure}
\begin{figure}[ht!] 
	\centering
	\includegraphics[width=\textwidth]{keff_zoomed.png}
	\caption{Zoomed effective multiplication factor for 150-EFPD time interval 
		(reproduced from Rykhlevskii \emph{et al.} 
		\cite{rykhlevskii_modeling_2019}).}
	\label{fig:keff_zoomed_msbr}
\end{figure}

First, Serpent calculates the effective multiplication factor for the 
beginning of the cycle (there is fresh fuel composition at the first step). 
Next, it computes the new fuel salt composition at the end of a 3-day 
depletion. The corresponding effective multiplication factor is much smaller 
than the previous one. Finally, Serpent calculates $k_{eff}$ for the depleted 
composition after applying feeds and removals. The $k_{eff}$ increases 
accordingly since major reactor poisons (e.g. Xe, Kr) are removed, while fresh 
fissile material ($^{233}$U) from the protactinium decay tank is added.  

Additionally, the presence of rubidium, strontium, cesium, and barium in the 
core are disadvantageous to reactor physics. 
Overall, the effective multiplication factor gradually decreases from 1.075 to 
$\approx$1.02 at equilibrium after approximately 6 years of irradiation. 

Loading initial fuel salt composition into the \gls{MSBR} core leads to a 
supercritical configuration (Figure ~\ref{fig:fp_removal}). After reactor 
startup, the effective multiplication factor for the case with volatile gas 
and noble metal removal is approximately 7500 pcm  higher than for the case 
with no fission product removal. This significant impact on the reactor core is
achieved due to immediate removal (20 sec cycle time) of elements with a high 
absorption cross section (e.g., Xe, Kr, Mo). The effect of rare earth 
element removal was considerable a few months after startup and reached 
approximately 5500 pcm after 10 years of operation. The rare earth elements 
were removed at a slower rate (50-day cycle time). Moreover, 
Figure~\ref{fig:fp_removal} demonstrates that batch-wise removal of strong 
absorbers every 3 days did not necessarily lead to fluctuation in results 
but rare earth element removal every 50 days caused an approximately 600 pcm 
jump in reactivity.

The effective multiplication factor of the core reduces gradually over 
operation time because the fissile material ($^{233}$U) continuously depletes 
from the fuel salt due to fission while fission products simultaneously
accumulate in the fuel salt. Eventually, without fission product removal, 
the reactivity decreases to the subcritical state after approximately 500 and 
1300 days of operation for cases with no removal and volatile gas \& noble 
metal removal, respectively. The time when the simulated core reaches 
subcriticality ($k_{eff}<$1.0) for the full-core model) is called the core 
lifetime. Therefore, removing fission products provides significant neutronic 
benefit and enables a longer core lifetime.
\begin{figure}[t] % replace 't' with 'b' to force it to 
	\centering
	\includegraphics[width=\textwidth]{keff_rem_cases.png} 
	\caption{Calculated effective multiplication factor for the full-core 
	\gls{MSBR} model with removal of various fission product groups over 10 
	years of operation (reproduced from Rykhlevskii \emph{et al.} 
	\cite{rykhlevskii_modeling_2019}).}
	\label{fig:fp_removal}
\end{figure}

These preliminary results have demonstrated SaltProc's capability to find 
the equilibrium fuel salt composition (where equilibrium is defined as when 
the number densities of major isotopes vary by less than 1\% over several 
years). Additionally these results showed the benefits of continuous fission 
product removal for a thermal \gls{MSR} design.
