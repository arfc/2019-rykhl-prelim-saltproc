\chapter[Introduction]{Introduction}

\section{Motivation}

Nowadays humankind has only few ways to generate reliable, nonintermittent base load power: fossil fuel, hydroelectric, geothermal, and nuclear energy. Because of increasing global warming and climate change concerns, sources that have negligible CO$_2$ footprints represent crucial measures for control global temperature change. From an environmental viewpoint hydro and nuclear power are preferable ways to generate reliable power. Nevertheless, potential for hydro power is strictly limited by local geographical conditions and significantly impacts the environment, affecting land use, homes, and natural habitats in the dam area, hence, the only one option left is nuclear power. Nuclear power plants generate 4.9\% of global energy production \cite{noauthor_key_2017}, a figure which is projected to stay constant up to 2040 while electricity demand is predicted to increase by 30\% \cite{noauthor_world_2017}. Unfortunately, because of concerns regarding safety, nuclear weapon prolifiration, radioactive waste treatment, and competitiveness with other sources of energy (i.e. renewables), a negative public attitude to nuclear has formed in many developed countries, which makes it challenging to advocate its zero emissions benefits.

\glsentryfirstplural{MSR} are among the six advanced reactor concepts that have been chosen by the Generation IV International Forum (GIF) for further research and development. \glspl{MSR} offer significant improvements ``in the four broad areas of sustainability, economics, safety and reliability, and proliferation resistance and physical protection" \cite{doe_technology_2002}. To achieve the goals formulated by the GIF, \glspl{MSR} attempt to simplify the reactor core and improve inherent safety by using liquid coolant which is also a fuel\footnote{Herein \glspl{MSR} are assumed to be reactors with liquid fuel which simultaneously serves as coolant.}. In the thermal spectrum \gls{MSR}, fluorides of fissile and/or fertile materials (i.e. UF$_4$, PuF$_3$ and/or ThF$_4$) are mixed with carrier salts to form a liquid fuel which is circulated in a loop-type primary circuit \cite{haubenreich_experience_1970}. This innovation leads to immediate advantages over traditional, solid-fueled, reactors. These include near-atmospheric pressure in the primary loop, relatively high coolant temperature, outstanding neutron economy, a high level of inherent safety, reduced fuel preprocessing, and the ability to continuously remove fission products and add fissile and/or fertile elements without shutdown \cite{leblanc_molten_2010}. The possibility of continuously removing neutron poisons 
allows significantly increase fuel burnup and thus improves the resource utilization 
of \glspl{MSR}. Finally, the MSR also could be employed as a converter reactor for transmutation of spent fuel from current \gls{LWR}.

Recently, interest in \glspl{MSR} has resurged, with multiple new companies 
pursuing commercialization of \gls{MSR} designs\footnote{Examples include 
liquid-fueled molten salt designs from Terrapower, Terrestrial, ThorCon, Flibe, Copenhagen Atomics, etc.}. China's \gls{MSR} program has been initiated in 2011 and promise 
to start-up 2MW$_{th}$ liquid-fueled test \gls{MSR} (TMSR-LF1) in 2020, 10MW$_{th}$ 
demonstration reactor (TMSR-LF2) in 2025 and GW level commercial reactor in 2050 \cite{zhang_review_2018}. European Union funds the Safety Assessment of the Molten Salt Fast Reactor (SAMOFAR) project, in which several European research institutes and universities are developing various molten salt reactor prototypes such as the \gls{MSFR} \cite{fiorina_molten_2013}, the \gls{MOSART} \cite{ignatiev_molten_2014}.
To further develop of these \gls{MSR} concepts, particularly with respect to their strategies for online reprocessing and refueling, computational analysis methods capturing their unique reactor physics and process chemistry are needed.

To further develop these \gls{MSR} concepts, particularly with respect to their  
strategies for online reprocessing and refueling, computational analysis methods capturing their unique reactor physics and process chemistry are needed.
However, most contemporary nuclear reactor physics software is unable to 
perform depletion calculations in an online reprocessing regime. To correctly 
determine properties and operation of \glspl{MSR}, including injection/removal 
schemes, computational tools to accurately model the processes which are unique 
for this particular reactor type. Specifically, the circulation of the
liquid fuel and the continuous feed or removal of elements (e.g. injection into 
and extraction from the fuel salt) are not in the focus of most contemporary 
nuclear reactor physics software, which were originally written primarily to 
simulate solid-fuel reactors. Thus, numerical simulation package that would be 
able model nuclear reactors with circulating fuel should be developed to close 
the gap.
%, including reactivity control by adjusting fissile material 
%injection, realistic, based on chemical processes, implementation of absorbers 
%removal, and taking into account non-instantaneous nature of injection/removal 
%processes and materials decay.

\section{Objectives}

\section{Methodology}