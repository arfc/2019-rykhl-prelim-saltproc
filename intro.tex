\chapter[Introduction]{Introduction}
\section{Motivation}
Humankind has only a few ways to generate reliable, nonintermittent 
baseload power: fossil fuels, hydro-power, geothermal power, and 
nuclear energy. Because of increasing global warming and climate 
change concerns, sources with negligible CO$_2$ footprints 
are crucial measures for global temperature change control. 
From an environmental viewpoint, hydro and nuclear power are 
preferable ways to generate reliable power. Nevertheless, the 
potential for a hydro-power is strictly limited by local geographical 
conditions; hence, the only option left is a nuclear power. Nuclear 
power plants generate 4.9\% of global energy \cite{noauthor_key_2017}. 
Moreover, a nuclear share in energy generation is projected to stay 
constant through 2040, while electricity demand will 
increase by 30\% \cite{noauthor_world_2017}. Unfortunately, a negative 
public attitude to nuclear was formed in many developed countries 
because of concerns regarding safety, nuclear weapons 
proliferation, radioactive waste treatment, and competitiveness with 
other sources of energy (i.e., renewables). This negative public 
attitude toward nuclear energy makes it challenging to justify its 
zero-emissions benefits.

Generation IV International Forum (GIF) chose \glspl{MSR} among the six 
advanced reactor concepts for further research and development. \glspl{MSR} 
offer significant improvements ``in the four broad areas of sustainability, 
economics, safety and reliability, and proliferation resistance and physical 
protection" \cite{doe_technology_2002}. To achieve the goals formulated by the 
GIF, \glspl{MSR} simplify the reactor core and improve inherent safety by 
using liquid coolant which is also a fuel\footnote{Herein \glspl{MSR} are 
assumed to be reactors with liquid fuel which simultaneously serves 
as coolant.}. In a thermal spectrum \gls{MSR}, liquid fuel is consists 
of carrier salt (i.e. LiF or LiF-BeF$_2$) and fluorides of fissile 
and/or fertile materials (i.e. UF$_4$, PuF$_3$ and/or ThF$_4$) 
which is circulates in a loop-type primary circuit 
\cite{haubenreich_experience_1970}. 
This innovation leads to immediate advantages over traditional, 
solid-fueled, reactors. These include near-atmospheric pressure 
in the primary loop, relatively high coolant temperature, outstanding 
neutron economy, a high level of inherent safety, reduced fuel 
preprocessing, the ability to continuously remove fission products 
and add fissile and/or fertile elements without shutdown 
\cite{leblanc_molten_2010}. The possibility of continuously removing 
neutron poisons increases the potential fuel burnup and thus 
improves the resource utilization of \glspl{MSR}. Finally, the \gls{MSR} 
also could be employed for transmutation of 
spent fuel from current \glspl{LWR} \cite{fratoni_design_2004}.

Recently, interest in \glspl{MSR} has resurged, with multiple new companies 
pursuing commercialization of \gls{MSR} designs\footnote{Examples 
include liquid-fueled \gls{MSR} designs from Terrapower, Terrestrial, 
ThorCon, Flibe, Copenhagen Atomics, Elysium, etc.}. China's \gls{MSR} program 
was initiated in 2011 and promises to startup a 2MW$_{th}$ 
liquid-fueled test \gls{MSR} in 2020, a 10MW$_{th}$ 
demonstration reactor in 2025, and a gigawatt-level 
commercial reactor in 2050 \cite{zhang_review_2018}. The European 
Union funds the Safety Assessment of the Molten Salt Fast Reactor 
(SAMOFAR) project, in which several European research institutes and 
universities are developing various molten salt reactor prototypes 
such as the \gls{MSFR} \cite{fiorina_molten_2013} and the \gls{MOSART} 
\cite{ignatiev_molten_2014}. To advance these \gls{MSR} concepts, particularly 
concerning their strategies for online reprocessing and refueling, 
we need computational analysis methods capturing their unique reactor physics, 
 fuel reprocessing mechanics and chemistry. 

The main objective of the proposed work is to develop the online 
reprocessing simulation package, SaltProc, which couples with the 
continuous-energy Monte Carlo depletion calculation code, Serpent 2 
\cite{leppanen_serpent_2014}, for liquid-fueled \gls{MSR} depletion 
simulations. The ultimate objective of this effort is to develop a generic 
open-source tool capable of simulating a wide range of liquid-fueled 
systems, including two-fluid, multi-region designs, and validate it against 
existing modeling efforts. 

This document outlines the motivation, preliminary work, and future work 
proposed towards developing a simulation tool for analyzing fuel depletion in 
a liquid-fueled \glspl{MSR}. Chapter 1 serves as a literature review, 
providing background on fuel burnup, online fuel reprocessing approaches, 
safety parameters evolution during the reactor operation, and how these 
concepts have been applied to a wide range of \glspl{MSR} in the literature. 
Chapter 2 covers modeling online reprocessing details and proposed computation 
tool architecture. Chapter 3 explains the \gls{VV} method, demonstration 
cases, safety parameters evolution. Specifically, these demonstration and 
verification efforts will focus on the \gls{TAP} \gls{MSR} because it well 
analyzed in the literature. Chapter 4 gives the safety parameters overview and 
outlines the plan for analyzing these parameters evolution during 
\gls{TAP} reactor lifetime. Finally, remaining future work and expected 
contributions to the nuclear community are summarized in Chapter 5.

\section{Fuel burnup and online reprocessing}
All liquid-fueled \gls{MSR} designs involve varying levels of online fuel 
processing. Minimally, volatile gaseous fission products (e.g., Kr, Xe) 
escape from the fuel salt during routine reactor operation and must be 
captured. Additional systems might be used to enhance the removal of those 
elements. Most designs also call for the removal of rare earth metals from 
the core since these metals act as neutron poisons. Some designs suggest a 
more elaborate list of elements to process (figure~\ref{fig:periodic_tab}), 
including the temporary removal of protactinium from the salt or other 
regulation of the actinide inventory in the fuel salt 
\cite{ahmad_neutronics_2015}. Fresh fuel salt with dissolved fissile and/or 
fertile material (e.g., $^{233}$U, $^{232}$Th, \gls{LEU}, a transuranic 
vector from \gls{LWR} \gls{SNF}) make-up the salt mass loss caused by poisons 
removal and conserves the total mass in the primary loop.
\begin{figure}[htp!] % replace 't' with 'b' to 
  \centering
  \includegraphics[width=0.9\textwidth]{periodic_map.png}
  \caption{Processing options for \gls{MSR} fuels (figure reproduced from 
  Ahmed \emph{et al.} \cite{ahmad_neutronics_2015}).}
  \label{fig:periodic_tab}
\end{figure}

Most of the liquid-fueled nuclear reactors adopted nonintermittent 
separations and feeds: the core material is moved to or from the core at all 
times (continuous) or specific intervals (batch). Contrarily, in a 
solid-fueled reactor, fission products and actinides remain within the initial 
fuel material during and after the operation until reprocessing. The ability 
to perform online fuel salt reprocessing improves the potential neutronics 
performance of liquid-fueled reactors. First, it is unnecessary for 
liquid-fueled reactors to operate with excess reactivity because fissile 
material is continuously being added to the core. Second, continuously 
removing fission products, including strong absorbers (poisons) should 
significantly improve fuel utilization and decrease parasitic neutron 
absorption. Finally, for a breeder (reactor with a \gls{CR}\footnote{\gls{CR} 
$\equiv$ fissile generated/fissile consumed: if CR $<$ 1, the reactor is a 
``converter''; CR $\equiv$ 1, an ``isobreeder''; CR $>$ 1, a 
``breeder.''}$>1$), excess of fissile material might be continuously evacuated 
from the core and used to startup new reactors. Nevertheless, the removal of 
each element from the liquid fuel salt presents a unique challenge in terms of 
chemical separation, storage, and disposal of the separated materials.

Continuous fuel salt reprocessing prevents the usage of most contemporary 
nuclear reactor fuel burnup software. Few code packages were developed 
specifically for \gls{MSR} depletion simulations at universities and 
research institutions and are not available for external use. The foundation 
for these in-house tools was based on early \gls{MSR} simulation methods at 
\gls{ORNL}, which integrated neutronic and fuel cycle codes (i.e., Reactor 
Optimum Design (ROD) \cite{bauman_rod_1971}) into operational plant tools 
(i.e., Multiregion Processing Plant (MRPP) \cite{kee_mrpp_1976}) for \gls{MSR} 
fuel reprocessing system design. A summary of recent efforts is listed in  
table~\ref{tab:msr_codes}.
\begin{table}[t]
\fontsize{9}{11}\selectfont
\caption{Tools and methods for liquid-fueled \glspl{MSR} fuel salt depletion 
analysis.}
\begin{tabularx}{\textwidth}{X X X X X} 
\hline 
&Nuttin \emph{et al.}, 2005 \cite{nuttin_potential_2005}& Aufiero \emph{et al.}, 
2013 \cite{aufiero_extended_2013} & Betzler \emph{et al.}, 2018 
\cite{betzler_fuel_2018}&Proposed work \\ [12pt]
\hline
Neutron transport software & \gls{MCNP} & Serpent 2 & SCALE6.2 & Serpent 2 \\ 
[12pt]
Neutron transport method & \multicolumn{2}{c}{Monte Carlo continuous energy} & 
Deterministic discrete ordinates & Monte Carlo continuous energy \\ [12pt]
Burnup software & REM (in-house) code & Serpent 2 & ORIGEN-S & Serpent 2 \\ 
[12pt]
Geometry model & three cells & full-core 3D & unit cell & full-core 3D\\ [12pt]
\gls{FP} removal/feed  & continuous &continuous & batch-wise & batch-wise\\ 
[12pt]
Separation efficiency &\multicolumn{3}{c}{fixed, must be determine by user 
before simulation} & variable of many parameters \\ [12pt]
Fuel reprocessing plant & \multicolumn{3}{c}{single component, ``black'' box 
model} & realistic multi-component model \\ [12pt]
Reactivity control & \multicolumn{2}{c}{continuous adjustment of fissile 
material injection} & batch injection of fissile material & periodical 
adjustment of geometry and fissile material injection\\
\hline
\end{tabularx}
  \label{tab:msr_codes}
\end{table}

Two main online reprocessing simulation approaches are commonly used in the 
literature: materials move to and from the core continuously or at specific 
time steps (batch-wise). In the batch-wise approach, the burnup simulation 
stops at a given time and restarts with a new liquid fuel composition 
(after removal of discarded materials and addition of fissile/fertile 
materials).

Recently, Nuttin \emph{et al.} developed in-house 
depletion code REM which directly couples with the \gls{MCNP}  
\cite{noauthor_mcnp_2004} to simulate fuel salt material evolution in 
simplified \gls{MSBR}-like reactor. That work directly integrated
Bateman differential equations using neutron flux from the \gls{MCNP}, 
tracking all the isotopes available in the data library, and control 
reactivity to maintain reactor critical \cite{nuttin_potential_2005}.

In a similar vein, Aufiero \emph{et al.} extended Serpent 2 for continuous 
reprocessing simulations by explicitly introducing ``reprocessing'' time 
constants into the system of Bateman equations and adding effective decay and 
transmutation terms for each nuclide \cite{aufiero_extended_2013}. The 
developed extension directly accounts for the effects of online fuel 
reprocessing on depletion calculations and features a reactivity control 
algorithm. The extended version of Serpent 2 was assessed against a dedicated 
version of the deterministic ERANOS-based EQL3D procedure in 
\cite{fiorina_investigation_2013} and applied to analyze the \gls{MSFR} fuel 
salt isotopic evolution.

\gls{ORNL} researchers have developed ChemTriton, a Python script for
SCALE/TRITON which uses the batch-wise approach to simulate a continuous 
reprocessing and refill for either single or multiple fluid designs. 
ChemTriton models salt treatment, separations, discharge, and refill using a 
unit-cell MSR SCALE/TRITON depletion simulation over small time steps to 
simulate continuous reprocessing and deplete the fuel salt 
\cite{powers_new_2013, betzler_fuel_2018}.

Most of the existing tools represented fuel salt reprocessing plant as an 
invariable ``black box'' model which removes target elements all at once with 
a fixed efficiency, determined by the user before starting the depletion 
simulation. Typical inputs and outputs for this ``black box'' model are 
vectors of elements and extraction efficiencies and can be expressed as 
follows:
\begin{equation}
\begin{bmatrix}
N^{in}_{0} \\ \vdots \\ N^{in}_{e} \\ \vdots \\ N^{in}_{E} \\
\end{bmatrix} 
\times
\begin{bmatrix}
\epsilon_{0} \\ \vdots \\ \epsilon_{e} \\ \vdots \\ \epsilon_{E} \\
\end{bmatrix} =
\begin{bmatrix}
N^{out}_{0}\\ \vdots \\ N^{out}_{e} \\ \vdots \\N^{out}_{E}  \\
\end{bmatrix}
\end{equation}
where $N^{in/out}$ is the number density of atoms and $\epsilon$ is the 
extraction efficiency for all elements $e$ in $(0, E)$. Main issues related 
with static ``black box'' model assumptions in the literature: 
\paragraph{Time-independent separation efficiency vector.} Realistically,   
	a long-term depletion simulation might require a time-dependent extraction 
	efficiency vector.
\paragraph{The separation efficiency independent of the reactor operational 
	parameters.} But the extraction efficiency depends on temperature, power 
	level, current fuel salt isotope composition, and material mass flow rate.
\paragraph{All reprocessing plant components are treated as a single ``black 
box'' component.} However, the fuel salt in a reprocessing plant undergoes 
many separate components (e.g., helium bubbling, nickel mesh filter, etc.) 
which target specific elements. Some of these components can be connected in 
series, parallel, or series-parallel. The ``black box'' model (only single 
process) requires massive pre-simulation analytic work from the user to 
calculate lumped separation efficiency vector before a simulation is run and 
cannot be adjusted during the simulation.	Finally, waste stream from each 
component cannot be tracked separately, which is necessary for for fuel 
reprocessing system optimization.

Some of the tools listed in table~\ref{tab:msr_codes} used major 
approximations that may lead to inaccurate fuel evolution predictions, and 
others are not available for external users. A purpose-made, open-source 
simulation package, SaltProc, which expands the capability of the 
continuous-energy Monte Carlo Burnup calculation code, Serpent 2, for 
simulation liquid-fueled \gls{MSR} operation is proposed in this work.

\subsection{Operational and safety parameters evolution}
In contrast with conventional solid-fueled reactors, which require a 
complete fuel replacement every 4-5 years\footnote{For the most common 
18-month cycle, during refueling personnel removing 1/3 of the fuel 
assemblies, re-arranging other assemblies, and loading fresh fuel into the 
core. Thus, each fuel assembly is kept in the core at most $3\times 18=54$ 
months.}, the initial fuel salt batch stays in the \gls{MSR} reactor primary 
loop during the whole lifetime. Therefore, the fuel salt accumulating 
\glspl{FP}, not captured by fuel reprocessing system, and transuranic 
elements\footnote{The chemical elements with atomic numbers greater than 
uranium (92).}. Continuous fuel salt composition degradation has a 
significant influence on the neutron energy spectrum and, consequently, 
affects the reactor behavior and raises significant safety concerns.

Nuttin {et al.} studied evolution of key safety parameter: temperature  
coefficient, and estimated it at start-up and at equilibrium state. The 
temperature coefficient quantify reactivity changes due to accidental 
temperature increase in the core and was calculated in that work as:
\begin{align}\label{eq:feedback}
& \qquad\qquad \alpha = \frac{k_{1200} - k_{900}}{\delta T} 
\intertext{where}
k_{900}, k_{1200}  &= \mbox{the multiplication coefficient for 900K and 
1200K} 
\nonumber \\
\delta T &= 1200K-900K \nonumber
\end{align}
That work showed negative and decreasing during reactor operation fuel  
temperature coefficient ($-1.5$ and $-1.0pcm/K$ at start-up and at equilibrium 
state, respectively); they also reported positive and time-invariant total 
temperature coefficient ($\approx+0.8pcm/K$) \cite{nuttin_potential_2005}. 
Recently, Park and colleagues expanded Nuttin' approach to a full-core 
high-fidelity \gls{MSBR} model and estimated safety parameters evolution over 
20 years of operation. These calculations showed relatively large negative 
total temperature coefficient during 20 years of the reactor operation; the 
coefficient magnitude weakens from $-3.21$ to $-1.41pcm/K$ at start-up and at 
equilibrium composition, respectively. Additionally, that work reported 
control rod effectiveness deterioration due to neutron spectrum hardening 
during reactor operation \cite{park_whole_2015}. 

More recently, Betzler \emph{et al.} reported key safety parameters evolution 
for the \gls{TAP} \gls{MSR}: the fuel reactivity coefficient at \gls{BOL} and 
15 years from \gls{BOL} is negative and decreasing slowly over the reactor 
lifetime; the moderator reactivity coefficient is small and positive at 
\gls{BOL} and became negative after 15 years of operation. Overall, thermal 
feedback seems to be stronger in the \gls{TAP} reactor and deteriorates 
insignificantly during the reactor operation. Notably, the authors ignored 
materials density change with temperature to simplify temperature 
coefficients calculation; thus, only  Doppler broadening has been taken into 
account. Finally, the researchers reported the total worth of all control rods 
in the \gls{TAP} core at start-up only \cite{betzler_assessment_2017}. 

The evolution of control rod worth in the \gls{TAP} was never reported in the 
literature before. The proposed work will illuminate the evolution of major 
safety parameters (fuel, moderator and total temperature coefficient, void 
reactivity coefficient, control rod worth) for the \gls{TAP} \gls{MSR} at 
various moments during the reactor operation. Additionally, the impact of 
neutron poisons accumulation (e.g., $^{135}$Xe) in the fuel salt during 
short-term transients (i.e., load following) on safety characteristics will be 
investigated.

\section{Chapter Conclusion}
The State-of-the-Art software packages for depletion analysis and evolution of 
safety parameters of liquid-fueled \gls{MSR} are reviewed in this Chapter. 
Based on this summary, I have identified a few possible directions for 
the improvement of the \gls{MSR} tools:
\paragraph{Reproducibility/code availability.} Only Serpent built-in online 
reprocessing feature is included in the distribution package and widely 
available for users; other mentioned tools are developed inside research 
institutions or universities and inaccessible for external users. In the era 
of GitHub \cite{github_github_2015} and international scientific 
collaboration, open and reproducible software practices must be implemented to 
resolve nuclear engineering simulation challenges presented by new reactor 
designs.
\paragraph{Realistic fuel reprocessing system model instead of ``black box''.} 
Major approximations in fuel reprocessing parameters deteriorates fuel salt 
composition predictions; the evolution of safety parameters accuracy is 
strongly dependent on fuel salt composition. Realistic fuel reprocessing 
system model will allow to optimize reprocessing component parameters, collect 
detailed data about waste streams, advance the reprocessing system design.
\paragraph{Variable extraction efficiency.} Most of the research efforts in 
the literature assumed ideal 100\% extraction efficiency, which stayed 
constant during the whole reactor lifetime. But realistically the efficiency 
is time-dependent and changes with respect to operational parameters: 
temperature, power level, salt composition, etc. Thus, the ability to set up 
dynamic separation efficiency must be added in \gls{MSR} simulation tools to 
advance depletion calculations.
\paragraph{Reactivity control.} Reconfigurable moderator configuration in the 
\gls{TAP} core presents a challenge because of the core geometry changes with 
respect to time. The reactivity control module which adjusts the core geometry 
to maintain criticality would be a great capability for simulating a new, more 
advanced \gls{MSR} concepts and short-term transients.
\paragraph{Safety characteristics evolution during reactor operation.} The 
\gls{MSR} fuel salt  accumulates \gls{FP} and transuranic elements which 
significantly shift neutron energy spectrum. Neutron energy hardening might 
worsen the core safety during operation. The impact of the fuel salt evolution 
on the \gls{MSR} safety parameters must be carefully investigated and reported.

The proposed work will hopefully overcome these issues and demonstrate the 
tool capabilities for promising \gls{TAP} \gls{MSR} concept.