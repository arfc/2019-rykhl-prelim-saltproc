\chapter[Introduction]{Introduction}

\section{Motivation}

Nowadays humankind has only few ways to generate reliable, nonintermittent base load power: fossil fuel, hydroelectric, geothermal, and nuclear energy. Because of increasing global warming and climate change concerns, sources that have negligible CO$_2$ footprints represent crucial measures for control global temperature change. From an environmental viewpoint hydro and nuclear power are preferable ways to generate reliable power. Nevertheless, potential for hydro power is strictly limited by local geographical conditions and significantly impacts the environment, affecting land use, homes, and natural habitats in the dam area, hence, the only one option left is nuclear power. Nuclear power plants generate 4.9\% of global energy production \cite{noauthor_key_2017}, a figure which is projected to stay constant up to 2040 while electricity demand is predicted to increase by 30\% \cite{noauthor_world_2017}. Unfortunately, because of concerns regarding safety, nuclear weapon prolifiration, radioactive waste treatment, and competitiveness with other sources of energy (i.e. renewables), a negative public attitude to nuclear has formed in many developed countries, which makes it challenging to advocate its zero emissions benefits.

\glsentryfirstplural{MSR} are among the six advanced reactor concepts that have been chosen by the Generation IV International Forum (GIF) for further research and development. \glspl{MSR} offer significant improvements ``in the four broad areas of sustainability, economics, safety and reliability, and proliferation resistance and physical protection" \cite{doe_technology_2002}. To achieve the goals formulated by the GIF, \glspl{MSR} attempt to simplify the reactor core and improve inherent safety by using liquid coolant which is also a fuel\footnote{Herein \glspl{MSR} are assumed to be reactors with liquid fuel which simultaneously serves as coolant.}. In the thermal spectrum \gls{MSR}, fluorides of fissile and/or fertile materials (i.e. UF$_4$, PuF$_3$ and/or ThF$_4$) are mixed with carrier salts to form a liquid fuel which is circulated in a loop-type primary circuit \cite{haubenreich_experience_1970}. This innovation leads to immediate advantages over traditional, solid-fueled, reactors. These include near-atmospheric pressure in the primary loop, relatively high coolant temperature, outstanding neutron economy, a high level of inherent safety, reduced fuel preprocessing, and the ability to continuously remove fission products and add fissile and/or fertile elements without shutdown \cite{leblanc_molten_2010}. The possibility of continuously removing neutron poisons 
allows significantly increase fuel burnup and thus improves the resource utilization 
of \glspl{MSR}. Finally, the MSR also could be employed as a converter reactor for transmutation of spent fuel from current \gls{LWR}.

Recently, interest in \glspl{MSR} has resurged, with multiple new companies 
pursuing commercialization of \gls{MSR} designs\footnote{Examples include 
liquid-fueled molten salt designs from Terrapower, Terrestrial, ThorCon, Flibe, Copenhagen Atomics, etc.}. China's \gls{MSR} program has been initiated in 2011 and promise 
to start-up 2MW$_{th}$ liquid-fueled test \gls{MSR} (TMSR-LF1) in 2020, 10MW$_{th}$ 
demonstration reactor (TMSR-LF2) in 2025 and GW level commercial reactor in 2050 \cite{zhang_review_2018}. European Union funds the Safety Assessment of the Molten Salt Fast Reactor (SAMOFAR) project, in which several European research institutes and universities are developing various molten salt reactor prototypes such as the \gls{MSFR} \cite{fiorina_molten_2013}, the \gls{MOSART} \cite{ignatiev_molten_2014}.
To further develop of these \gls{MSR} concepts, particularly with respect to their strategies for online reprocessing and refueling, computational analysis methods capturing their unique reactor physics and process chemistry are needed.

To further develop these \gls{MSR} concepts, particularly with respect to their  
strategies for online reprocessing and refueling, computational analysis methods capturing their unique reactor physics and process chemistry are needed.
However, most contemporary nuclear reactor physics software is unable to 
perform depletion calculations in an online reprocessing regime. To correctly 
determine properties and operation of \glspl{MSR}, including injection/removal 
schemes, computational tools to accurately model the processes which are unique 
for this particular reactor type. Specifically, the circulation of the
liquid fuel and the continuous feed or removal of elements (e.g. injection into 
and extraction from the fuel salt) are not in the focus of most contemporary 
nuclear reactor physics software, which were originally written primarily to 
simulate solid-fuel reactors. Thus, numerical simulation package that would be 
able model nuclear reactors with circulating fuel should be developed to close 
the gap.
%, including reactivity control by adjusting fissile material 
%injection, realistic, based on chemical processes, implementation of absorbers 
%removal, and taking into account non-instantaneous nature of injection/removal 
%processes and materials decay.

\section{Literature review}
Most contemporary nuclear reactor physics software is unable to perform depletion calculations in an online reprocessing regime. Furthermore, no established tool for liquid-fueled \gls{MSR} neutronics and fuel cycle evaluation exist, except internally developed tools from universities and research institutions can approximate online refueling. The foundation for these tools was based on early \gls{MSR} simulation methods at \gls{ORNL}, which integrated neutronic and fuel cycle codes (i.e., Reactor 
Optimum Design (ROD) \cite{bauman_rod_1971}) into operational plant tools (i.e., 
Multiregion Processing Plant (MRPP) \cite{kee_mrpp_1976}) for \gls{MSR} and 
reprocessing system design. Based on 
this approach, recent tools from universities and research institutions can
approximate online refueling \cite{serp_molten_2014-1}. A summary of recent
efforts is listed in table~\ref{tab:fs_codes}.
\begin{table}[ht!]
\caption{Tools and methods for liquid-fueled \glspl{MSR} fuel salt depletion analysis.}
\begin{tabularx}{\textwidth}{ m | b | x } 
\hline Neutronic code    & \qquad Authors & Spectrum   \\
\hline
\gls{MCNP}/REM \cite{noauthor_mcnp_2004,heuer_simulation_2010}  & Doligez 
\emph{et al.}, 2014; Heuer \emph{et al.}, 2014  
\cite{doligez_coupled_2014,heuer_towards_2014}    & fast \\
\hline
ERANOS \cite{ruggieri_eranos_2006}  & Fiorina \emph{et al.}, 2013 
\cite{fiorina_investigation_2013}            & fast \\
\hline
KENO-IV/ORIGEN \cite{goluoglu_monte_2011,gauld_isotopic_2011}     & Sheu 
\emph{et al.}, 2013 \cite{sheu_depletion_2013} & fast \\
\hline
SERPENT2 \cite{leppanen_serpent_2015}  & Aufiero \emph{et al.}, 2013 
\cite{aufiero_extended_2013}; Ashraf \emph{et al.}, 2018 \cite{ashraf_nuclear_2018} & fast \\
\hline
DIF3D \cite{derstine_dif3d_1984} & Zhou \emph{et al.}, 2018 
\cite{zhou_fuel_2018-1} & thermal/ fast \\
\hline
\gls{MCNP}/REM  & Nuttin \emph{et al.} \cite{nuttin_potential_2005}&thermal  \\ 
\hline
MCODE/ORIGEN2 \cite{xu_mcode_2008,croff_users_1980}      & Ahmad \emph{et al.}, 
2015 \cite{ahmad_neutronics_2015}   & thermal  \\
\hline
OpenMC/ORIGEN-S \cite{romano_openmc_2015-1,rearden_scale_2018}  & de Lanversin \emph{et al.}, 
2017 \cite{de_troullioud_de_lanversin_toward_2017}   & thermal  \\
\hline
\gls{MCNP}6/CINDER90 \cite{goorley_mcnp6_2013}     & Park \emph{et al.}, 2015; 
Jeong \emph{et al.}, 2016 \cite{park_whole_2015, jeong_equilibrium_2016}& 
thermal\\
\hline
SCALE/TRITON \cite{bowman_scale_2011,powers_new_2013}    & Powers \emph{et al.}, 
2014; Betzler \emph{et al.}, 2017 
\cite{powers_new_2013,powers_inventory_2014,betzler_molten_2017}& thermal/ fast\\
\hline
SERPENT2     & Rykhlevskii \emph{et al.}, 2017-2019 \cite{rykhlevskii_online_2017, rykhlevskii_full-core_2017, rykhlevskii_advanced_2018,rykhlevskii_modeling_2019} & 
thermal\\
\hline
\end{tabularx}
  \label{tab:fs_codes}
\end{table}
\FloatBarrier

References \cite{de_troullioud_de_lanversin_toward_2017,doligez_coupled_2014,heuer_towards_2014, sheu_depletion_2013, aufiero_extended_2013} provide some form of reactivity control, 
and methods \cite{doligez_coupled_2014,heuer_towards_2014,aufiero_extended_2013,ahmad_neutronics_2015, 
park_whole_2015,jeong_equilibrium_2016,rykhlevskii_modeling_2019,nuttin_potential_2005} use a set of all nuclides in depletion calculations. 

\subsection{Batch-wise online reprocessing approach}
Many liquid-fueled \gls{MSR} designs rely on online fuel processing in which  
material moves to and from the core continuously or at specific time steps 
(batch-wise). In the batch-wise approach, the burn-up simulation stops at a given 
time and restarts with a new liquid fuel composition (after removal of discarded 
materials and addition of fissile/fertile materials). \gls{ORNL} researchers 
have developed ChemTriton, a Python-based script for SCALE/TRITON which uses the 
batch-wise approach to simulate a continuous reprocessing and refill for either single 
or multiple fluid designs. ChemTriton models salt 
treatment, separations, discharge, and refill using a unit-cell \gls{MSR} 
SCALE/TRITON depletion simulation over small time steps to simulate continuous 
reprocessing and deplete the fuel salt \cite{powers_new_2013}. Methods listed in 
references \cite{zhou_fuel_2018-1,sheu_depletion_2013,park_whole_2015,jeong_equilibrium_2016, powers_inventory_2014,betzler_molten_2017,rykhlevskii_modeling_2019} 
as also employ a batch-wise approach. This approach has a few disadvantages. First, 
this method assumes that material transfer from the reactor primary loop to the reprocessing 
plant is discrete and instantaneous. In reality, some small fraction of fuel salt \textbf{continuously} flowing to reprocessing plant and approximately the same 
\textbf{continuous} material flow returning back to keep mass balance in a 
primary loop. Second, this approach assumes that material accumulation in the core 
during the time between separations or feeds does not affect reactor physics which 
is not true for relatively long (few days) depletion simulation time steps. These assumptions for simulation of the nonstop online reprocessing process tends to
 misrepresent depleted material composition obtained from simulation.

\subsection{Continuous online reprocessing approach}
Accounting for continuous removal or addition presents a greater challenge since it 
requires adding a term to the Bateman equations. In SCALE/TRITON, ORIGEN \cite{gauld_isotopic_2011} solves a set of Bateman equations using one-group averaged fluxes and cross-sections obtained from a transport calculation. Bateman equations that describe the rate of change of the isotopes due to neutron induced reactions and decay
processes could be written in this form \cite{aufiero_extended_2013}:

\begin{align}
        \frac{dN_i}{dt} &= \bar{\Phi}\sum\limits_{j}N_{j}\sigma_{j \rightarrow 		i} - \bar{\Phi}\sum\limits_{j}N_{i}\sigma_{i \rightarrow j} + \sum					\limits_{j}	N_{j}\lambda_{j}b_{j \rightarrow i} - N_{i}\lambda_{i}
\label{eq:bateman}
	\intertext{where} 
	N_i &= \mbox{number density of isotope i} \\
	N_j &= \mbox{number density of isotope j} \\
	\bar{\Phi} &= \mbox {average in the space and energy neutron flux} \\
	\sigma_{j \rightarrow i} &= \mbox{microscopic one-group transmutation cross section} \\
	\lambda_i &= \mbox{decay constant of nuclide i} \\
	\lambda_j &= \mbox{decay constant of nuclide j} \\
	b_{j \to i} &= \mbox{branching fractions of radioactive decay from nuclide j}
\end{align}

The four terms on the right-hand side of the equation represent (1) the production rate of nuclide $i$ from irradiation, (2) the loss rate of nuclide $i$ due to irradiation, (3) the decay rate of nuclide $j$ into nuclide $i$, and (4) the loss rate of nuclide $i$ due to decay. Mentioned earlier deterministic codes SCALE/TRITON and Monte Carlo codes MCNP, Shift, KENO-VI do not support non-zero removal or feeds rates for depletion simulations.

Online fuel reprocessing can be explicitly introduced in the system of equations by adding effective decay and transmutation terms for the various nuclides. Fuel refill during fuel salt composition evolution calculations could be taken into account by
adding a gain term on the right-hand side of the Bateman equation:
\begin{align*}
\bar{\Phi}\sum\limits_{k=^{235}U}N_{k}\sigma_{k,c}
\end{align*}
where $\sigma_{k,c}$ is the one-group capture cross section of fissile isotope (e.g. uranium-235).

The removal of fission products and protactinium is achieved by adding an explicit decay term to the Bateman equations. For the generic fission product, l, loss term can be added:
\begin{align*}
- N_{l}\lambda_{l,reproc}
\end{align*}
where $\lambda_{l,reproc}$ is the effective removal time constant of the particular chemical species. Fiorina \emph{et al.} simulated 
\gls{MSFR} depletion with continuous fuel salt reprocessing via introducing 
``reprocessing'' time constants into the ERANOS transport code 
\cite{fiorina_investigation_2013}. 
The latest SCALE release will also have the same functionality using truly 
continuous removals \cite{betzler_implementation_2017}. 
A similar approach is adopted to model true continuous feeds and removals using the MCNP transport code listed in references \cite{doligez_coupled_2014,heuer_towards_2014,nuttin_potential_2005}. 

This approach was recently implemented as a purpose-made extension within the continuous-energy Monte Carlo reactor physics and burn-up code SERPENT \cite{aufiero_extended_2013} 
using a described earlier methodology by explicitly introducing 
continuous reprocessing in the system of Bateman equations and adding effective 
decay and transmutation terms for each nuclide 
\cite{aufiero_extended_2013}. This was employed to study the material isotopic evolution of the 
\gls{MSFR}\cite{aufiero_extended_2013}. The developed extension directly accounts for the effects of online fuel 
reprocessing on depletion calculations and features a reactivity control 
algorithm. The extended version of SERPENT2 was assessed against a dedicated 
version of the deterministic ERANOS-based EQL3D procedure in
\cite{ruggieri_eranos_2006, fiorina_investigation_2013} and adopted to analyze 
the \gls{MSFR} fuel salt isotopic evolution. 

We employed this built-in SERPENT2 feature for a simplified unit-cell geometry of 
the thermal spectrum thorium-fueled \gls{MSBR} and found it unusable\footnote{ 
Some challenges in no particular order: mass conservation is hard to achieve; 
three types of mflow cards (0, 1 or 2) are indistinguishable in purpose; an 
unexplained difference between CRAM and TTA results; etc.}. Primarily,
it is undocumented, and the discussion forum for SERPENT users is the only useful 
source of information at the moment. Additionally, the reactivity control module described in Aufiero \emph{et al.} is 
not available in the latest SERPENT 2.1.30 release. Third, the infinite multiplication 
factor behavior for simplified unit-cell model obtained using SERPENT2 built-in 
capabilities \cite{rykhlevskii_online_2017} does not match with exist MCNP6/Python-script
 results for the similar model by Jeong and Park\footnote{ In our 
study k$_{\infty}$ drops from 1.05 to 1.005 during a 1200 days of depletion simulation 
while in Jeong and Park work this parameter decreasing slowly from 1.065 to 1.05 for the 
similar time-frame.} \cite{jeong_equilibrium_2016}. 

If these challenges can be overcome through verification against
ChemTriton/SCALE as well as this work (the SaltProc/SERPENT2 package), we hope to 
employ this SERPENT2 feature 
for removal of fission products with shorter residence time (e.g., Xe, Kr), 
since these have a strong negative impact on core lifetime and breeding 
efficiency.

Nuttin \emph{et al.} broke up the reactor core geometry into three \gls{MCNP} cells: 
one for salt channels, one for the salt plena above and below the core, and a 
third cell for the annulus. Consequently, the two-region reactor core was 
approximated by one region with averaged fuel/moderator ratio 
\cite{nuttin_potential_2005}.  Powers \emph{et 
al.}, Betzler \emph{et al.}, and Jeong \emph{et al.} 
\cite{powers_new_2013,powers_inventory_2014,betzler_modeling_2016, 
betzler_molten_2017, jeong_development_2014, jeong_equilibrium_2016} used a 
similar approach. This approach 
misrepresents the two-region breeder reactor concept. The unit-cell or one-region 
models may produce reliable results for homogeneous reactor cores (i.e. 
\gls{MSFR}, \gls{MOSART}) or for one-region single-fluid reactor designs (i.e. 
\gls{MSRE}). However, a two-region \gls{MSBR} must be simulated using a whole-core 
model to capture different neutron transport characteristics in the inner and 
outer regions of the core. In particular, most fissions happen in the inner 
region while breeding occurs in the outer zone. 

The present work introduces the online reprocessing simulation package, SaltProc, 
which expands the capability of the continuous-energy Monte Carlo Burnup 
calculation code, SERPENT2 \cite{leppanen_serpent_2015}, for simulation 
liquid-fueled \gls{MSR} operation 
\cite{rykhlevskii_modeling_2019}. It also reports the 
application of the coupled SaltProc-SERPENT2 system to the \gls{MSBR}, an 
extension of the work presented in 
\cite{rykhlevskii_full-core_2017, rykhlevskii_online_2017}. In this work, we 
analyzed \gls{MSBR} neutronics and fuel cycle to establish its equilibrium core 
composition. Additionally, we compared predicted operational and safety parameters of the \gls{MSBR} at 
both the initial and equilibrium states to characterize the evolution of its 
safety case over time. Finally, these simulations determined the appropriate $^{232}$Th feed rate  
for maintaining criticality and enabled analysis of the overall \gls{MSBR} fuel 
cycle performance.

The works described in \cite{park_whole_2015} and \cite{jeong_equilibrium_2016} 
are most similar to the work presented in this paper. However, a few major 
differences follow: (1) Park \emph{et al.} employed MCNP6 for depletion 
simulations while this work used SERPENT2; (2) the full-core reactor 
geometry herein is more detailed \cite{rykhlevskii_full-core_2017}; (3) Park \emph{et al.} and Jeong \emph{et al.} 
both only considered volatile gas removal, noble metal removal, and $^{233}$Pa separation while 
the current work implemented the more detailed reprocessing scheme specified in 
the conceptual \gls{MSBR} design \cite{robertson_conceptual_1971}; (4) the $^{232}$Th neutron 
capture reaction rate has been investigated to prove advantages of two-region core 
design; (5) the current work explicitly examines the independent impacts of removing specific fission product groups. 

The complex \gls{MSBR} geometry is challenging to describe in software input, 
and usually researchers make significant geometric simplifications to model it 
\cite{park_whole_2015}. This study leverages extensive computational 
resources to avoid these geometric approximations in order to accurately capture 
breeding behavior. 

\section{Objectives}

\section{Methodology}

\section{Outline}

