\chapter[Safety analysis]{Safety analysis}

\section{Safety and operational parameters}

\subsection{Temperature coefficient of reactivity}
The main physical principle underlying the reactor temperature feedback is an 
expansion of heated material. When the fuel salt temperature increases, the 
density of the salt decreases, but at the same time, the total volume of fuel 
salt in the core remains constant because it is bounded by the vessel. When 
the moderator rods temperature increases, the density of zirconium hydride 
decreases, creating additional space for fuel salt. 

Chapter 2, equation~\ref{eq:feedback} defined Total or Isothermal Temperature 
Coefficient (ITC) which expresses the dependence of the core reactivity on the 
combined effects of fuel and moderator temperature. But fuel and moderator 
temperature are rarely equal because fuel heats up much faster than moderator; 
thus, the fuel temperature coefficient (FTC) and the moderator temperature 
coefficient (MTC) should be calculated separately. In the base 
case, the fuel salt and the moderator temperatures are fixed at 900K which is  
operational temperature in the core. To determine the fuel salt temperature 
coefficient (FTC), I will perturbing only the fuel salt temperature to 800K 
and 1000K while fixing the moderator temperature at 900K (base case).  
Likewise, the moderator temperature coefficient (MTC) would be found by 
perturbing the moderator temperature from 900K to 800K and 1000K, while the 
fuel temperature is fixed at 900K. 

The range of temperature perturbation for the TC calculation has been selected 
based on operational parameters. The \gls{TAP} \gls{MSR} operates between 773 
and 973K (500-700$^o$C), which is far below the salt boiling point of 
approximately 1473K. The salt freezes if temperature drops below 773K. At the 
other end of the temperature spectrum, temperature greater than 973K passively 
melt a freeze plug, which drains the fuel salt from the reactor vessel to the  
drain tanks. The drain tanks has subcritical configuration with a high free 
surface area to readily dissipate heat by passive cooling 
\cite{transatomic_power_corporation_technical_2016}. Thus, calculating 
temperature coefficients in temperature range from 800 to 1000K seems to be 
conservative enough to capture outcome most of the accident transients.

Fuel and Moderator Temperature Coefficients for the temperature in range 
from 800 to 900K would be useful for analyzing following transients 
related with sudden fuel salt cooling: 
\begin{itemize}
	\item increase in heat removal by the secondary system;
	\item increase in the fuel salt flow rate;
	\item planned reactor shutdown.
\end{itemize}
Temperature coefficients for the temperature in range from 900 to 1000K 
would be used for transients related with the salt overheat: 
\begin{itemize}
	\item loss-of-coolant accident (LOCA) ;
	\item loss-of-flow accident (LOFA);
	\item loss of ultimate heat sink;
	\item  station blackout (SBO).
\end{itemize}
Thus, temperature coefficients will be calculated separately for heating up 
and cooling. My dissertation will include following  cases: 
\begin{enumerate}
	\item FTC (moderator rods temperature is fixed at 900K):
		\begin{enumerate}[label=(\alph*)]
			\item temperature of the salt rising from 900 to 1000K;
			\item temperature of the salt decreasing from 900 to 800K.
		\end{enumerate}
	\item MTC (the fuel salt temperature is fixed at 900K): 
		\begin{enumerate}[label=(\alph*)]
			\item temperature of the moderator rising from 900 to 1000K;
			\item temperature of the fuel declining from 900 to 800K.
		\end{enumerate}
	\item ITC: 
		\begin{enumerate}[label=(\alph*)]
			\item whole reactor temperature increasing from 900 to 1000K;
			\item whole reactor temperature decreasing from 900 to 800K;
		\end{enumerate}
\end{enumerate}
In the first case, changes in the fuel temperature will impact cross section 
temperature (Doppler broadening) and fuel density but the geometry is 
unchanged because the fuel is a liquid. The density of fuel salt 
changes with respect to temperature as follows \cite{janz_molten_1974}:
\begin{align}\label{eq:salt-den}
\rho_{salt}(T[K]) &= 6.105 - 12.720\times10^4 T \quad [g/cm^3]
\end{align}
In contrast, when the moderator temperature changing, the density, cross 
section temperature, and the geometry are changing also due to thermal 
expansion of the solid zirconium hydride (ZrH$_{1.66}$) rods. Accordingly, the 
new moderator density and sizes will be calculated using a linear temperature 
expansion coefficient of $2.734\times10^{-5}$K$^{-1}$ 
\cite{yamanaka_thermal_1999}. A new geometry input for Serpent 2, which takes 
into account displacement of the moderator  surfaces, will be created based on 
this information. Finally, temperature coefficient for each case will be 
calculated separately as follows:
\begin{align}
\alpha &= \frac{k_{eff}(T_{i+1}) - k_{eff}(T_i)}{k_{eff}(T_{i+1}) 
k_{eff}(T_{i}) (T_{i+1} - T_i)}
\intertext{where}
k_{eff} &= \mbox{effective multiplication factor} \nonumber \\
T_i &= \mbox{fuel salt temperature in (800K, 1000K)} \nonumber
\end{align}
By propagating the $k_{eff}$ statistical error provided by Serpent 2, 
uncertainty for each temperature coefficient will be obtain using formula:
\begin{align}
\delta\alpha &= \abs{\frac{1}{T_{i+1} - T_i}} \sqrt{\frac{\delta 
k_{eff}^2(T_{i+1})}{k_{eff}^4(T_{i+1})}  
+ \frac{\delta k_{eff}^2(T_i)}{k_{eff}^4(T_i)}}
\intertext{where}
\delta k_{eff} &= \mbox{statistical error for $k_{eff}$ from Serpent output} 
\nonumber
\end{align}
Notably, other sources of uncertainty are neglected, such as cross section 
measurement error and approximations inherent in the equations of state 
providing both the salt and moderator density dependence on temperature. 

\subsection{Reactivity control system rod worth}
In the \gls{TAP} concept control rods perform two main functions: to shutdown 
the reactor at any point during operation by introducing sufficient negative 
reactivity, and to control excess of reactivity after moderator rod 
reconfiguration during regular maintenance. In an accident, the control rods 
would be dropped down into the core. The reactivity worth of the control rods  
will be calculated for various positions to separately estimate the worth of 
each control rod, and the whole reactivity control system. Finally, control 
rod with the maximum worth will be localized to conduct basic safety test: at 
\gls{BOL} the reactor should not startup if a single rod (maximum worth rod) 
is accidentally ejected from the \gls{TAP} core.

The reactivity worth of the single control rod is defined as:
\begin{align}
& \qquad\qquad\qquad CRW = (k_{eff}^W - k_{eff}^I)\times 10^5 \;\; [pcm]
\intertext{where}
k_{eff}^W &= \mbox{effective multiplication factor when control rod is fully 
withdrawn} \nonumber \\
k_{eff}^I &= \mbox{effective multiplication factor when control rod is fully 
inserted} \nonumber 
\end{align}
The statistical error of the  reactivity worth will be obtain using formula:
\begin{align}
\delta CRW &= \sqrt{(\delta k_{eff}^W)^2 + (\delta k_{eff}^I)^2}
\intertext{where}
\delta k_{eff}^{W}, \delta k_{eff}^{I}, &= \mbox{$k_{eff}$ statistical error 
from Serpent output} 
\nonumber
\end{align}

\subsection{Axial Offset} \label{sec:axial-offset}
\gls{AOA} refers to a neutron flux depression in the top of a nuclear reactor 
core, which complicates the reactor operation. This problem occurs in  
\gls{PWR} plants and leads to variety of problems: increased local power 
peaking factors, lower that expected burnup and decreased control rod worth.
The Axial Offset is defined as:
\begin{align}
& \qquad\qquad\qquad A/O  = \frac{p_{top} - p_{bottom}}{p_{top} + p_{bottom}}
	\intertext{where}
p_{top}, p_{bottom} &= \mbox{fraction of rated power in a top and bottom half 
of the core} \nonumber
\end{align}

For the case of the \gls{TAP} \gls{MSR}, off-gas system components (e.g., 
sparger, entrainment separator) are introducing small bubbles of an inert gas 
(helium) into the fuel salt during operation. These helium bubbles when enters 
into the core would introduce unpredictable reactivity. Moreover, the diameter 
of the bubbles would raise from the bottom to the top of the core because of 
approximately 140$^{o}$C temperature difference between the reactor inlet and 
outlet. To take into account effect of the gas presence in the fuel salt, I 
will split the reactor core model into few axial regions with different 
corresponding salt density (figure~\ref{fig:axial-offset}).

The ideal gas law must be employed to find density of the fuel salt with gas 
bubbles\footnote{Assuming we know the bubble number and size from the 
component design.} 
in it:
\begin{align}
PV &= nRT 
\intertext{where}
P &= \mbox{pressure of the gas} \nonumber \\
V &= \mbox{volume of the gas} \nonumber \\
n &= \mbox{number of moles of the gas} \nonumber \\
R &= 8.31 \frac{J}{K\cdot mol} \nonumber \\
T &= \mbox{temperature of the gas} \nonumber
\end{align}
Assuming pressure in the vessel maintained fixed and number of moles of the 
helium is constant, the helium volume change related with temperature change  
can be defined as:
\begin{align}\label{eq:he-vol-change}
&\qquad\qquad\qquad V^{He}_i = \frac{V^{He}_0}{T_0} T_i
\intertext{where}
V^{He}_0, V^{He}_i &= \mbox{total He volume in the salt with temperature 
T$_0$, 
	T$_i$} \nonumber \\
T_0, T_i &= \mbox{temperature of the salt in a lower and i$^{th}$ axial layer} 
\nonumber
\end{align}
\begin{figure}[bth!] % replace 't' with 'b' to 
	\includegraphics[width=\textwidth]{axial_offset.png}
	\caption{Preliminary schematic view showing the \gls{TAP} model divided to 
		multiple axial layers with different densities of the salt. Assumed 
		linear 
		temperature growth from the bottom to the top.}
	\label{fig:axial-offset}
\end{figure}

Using equations~\ref{eq:salt-den},\ref{eq:he-vol-change}, density of the  
salt/helium mixture in each axial layer ($\rho_i$) will be calculated. 
Finally, Serpent 2 calculation will be performed for the model with 
non-uniform axial density distribution in the fuel salt to determine axial 
neutron flux distribution and axial offset (A/O) in the \gls{TAP} core.

\section{Preliminary results}

\subsection{Temperature coefficients and rod worth at BOL}
Table~\ref{tab:tcoef} summarizes temperature effects on reactivity in the 
\gls{TAP} core calculated for initial fuel salt composition. The fuel 
temperature coefficient is $-0.693$ and $-0.116$ pcm $\Delta k/k\cdot K^{-1}$ 
when perturbing the salt temperature from 900K to 800K and from 900K to 1000K, 
respectively. The MTC and ITC is negative and relatively large for both 
cases. All three temperature coefficients when perturbing from 900K to 
1000K did not matched coefficients when perturbing from 900K to 800K because 
cross sections are nonlinear with respect to temperature.

%%%%%%%%%%%%%%%%%%%%%%%%%%%%%%%%%%%%%%%%
\begin{table}[ht!]
	\caption{Temperature coefficients for the \gls{TAP} reactor at startup for 
	different temperature perturbations.}
	\begin{tabularx}{\textwidth}{ X  r r } \hline
		Reactivity coefficient     & From 900 to 800K [pcm/K] & From 900 to 
		1000K [pcm/K]                  \tabularnewline [5pt] \hline
		Fuel salt Temperature Coefficient (FTC) &
		$-0.693\pm9.25\times10^{-2}$ & $-0.116\pm9.33\times10^{-2}$ 
		\tabularnewline [5pt] \hline
		Moderator (ZrH$_{1.66}$) Temperature Coefficient (MTC) & 
		$-1.106\pm9.25\times10^{-2}$ & $-1.195\pm9.27\times10^{-2}$  
		\tabularnewline [5pt] \hline
		Isothermal Temperature Coefficient (ITC)  & 
		$-1.768\pm9.17\times10^{-2}$ &  $-1.301\pm9.27\times10^{-2}$  
		\tabularnewline [5pt] \hline
	\end{tabularx}
	\label{tab:tcoef}
\end{table}
%%%%%%%%%%%%%%%%%%%%%%%%%%%%%%%%%%%%%%%%%%%%%%%%%%%%%%%%%%%%%%%%%%%%%%%%%%%%%%%%
These reactivity coefficients will change most likely during operation due to 
changes in the neutron spectrum, \gls{SVF}, core geometry, amount of fuel in 
the core and fuel composition. The FTC would be expected to be more negative 
during operation as the spectrum thermalizes and additional, non-removing 
fission products and actinides build up in the fuel salt. Temperature 
coefficients should be repeated at middle-of-life and end-of-life to capture 
changes in these characteristics.

A configuration of 25 control rods (Figure~\ref{fig:tap-serpent-plan}) 
provides a reactivity worth of $1110\pm9.7$ pcm (1.1\%) at \gls{BOL}. The 
control rods worth would be expected to become more negative during operation 
because spectrum thermalizes and absorbing material (gadolinium) absorption 
cross section is larger in a thermal energy range. At the same time, that 
effect would be counteracted by the actinides (particularly, plutonium) 
accumulation in the core. Thus, control rod worth change during the reactor 
operation must be estimated.
