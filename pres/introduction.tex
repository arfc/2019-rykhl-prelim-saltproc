\subsection{Molten Salt Reactors}

\begin{frame}
\frametitle{Potential Generation IV reactor systems \cite{abram_generation-iv_2008}}
\begin{figure}[t]
	\vspace*{-0.1in}
	\includegraphics[height=0.7\textwidth]{./images/6_types.png}
\end{figure}            
\end{frame}


\begin{frame}
\frametitle{MSR (Molten Salt Reactor) types}
\begin{overlayarea}{\linewidth}{20\baselineskip}
\begin{block}{Stationary Fuel}<1-4>
	\begin{enumerate}
		\item Graphite block with TRISO fuel, clean salt works as 
		coolant (Fluoride-Salt-Cooled High-Temperature 
		Reactor (FHR))
		\item Plate Fuel: hexagonal fuel assembly is similar in shape to a typical sodium-cooled reactor
		\item Fuel Inside Radial Moderator (FIRM)
		\item Liquid fuel salt inside fuel rods cooled by clean salt 
		(Moltex Stable Salt Reactor)
	\end{enumerate}
\end{block}

\begin{block}{Mobile Fuel}<2-4>
	\begin{enumerate}
		\item<2-4> Mobile solid fuel elements (pebbles) cooled by 
		clean salt (PB-FHR)
		\item<3-4> Non-circulating liquid fuel salt (TerraPower \gls{MCFR}) 
		\item<4> \textbf{Circulating liquid fuel salt} which also works 
		as coolant (\gls{MSBR})
	\end{enumerate}
\end{block}
\end{overlayarea}
\end{frame}

\begin{frame}
\frametitle{Stationary and Mobile Solid fuel}
\vspace*{-0.1in}
\begin{figure}[t]
	\hspace*{-0.35in}
	\includegraphics[height=0.63\textwidth]{./images/solid_fuel.jpg}
	\caption{TRISO fuel particle (top) and FHR fuel designs (bottom). Source \cite{forsberg_basis_2016}.} 
\end{figure}   
\end{frame}

\begin{frame}
\frametitle{Mobile, Non-Circulating, Liquid Fuel}
\begin{figure}[t]
\vspace*{-0.1in}
\hspace*{-0.35in}
\includegraphics[height=0.6\textwidth]{./images/mcfr-crossection.jpg}
\caption{The TerraPower MCFR is an example of reactor design with \textbf{liquid, mobile, non-circulating} chloride salt fuel. Source \cite{doene_southern_2018}.}
\end{figure}   

\end{frame}

\begin{frame}
\frametitle{Mobile, Circulating, Liquid Fuel}
\begin{figure}[t]
\includegraphics[height=0.58\textwidth]{./images/msbr_scheme.png}
\caption{The \gls{MSBR} is an example of reactor design with \textbf{liquid, mobile, circulating} fluoride salt fuel \cite{rosenthal_molten-salt_1970}.}
\end{figure}   

\end{frame}

\subsection{Motivation}


\begin{frame}
\frametitle{Why Molten Salt Reactors with circulating fuel?}
\begin{block}{Liquid-fueled MSR 
		designs have following \textbf{potential} advantages:}
	\begin{enumerate}
		\itemsep1em
		\item High coolant temperature (600-750$^{\circ}$C) 
		$\Rightarrow$ potentially high thermal efficiency, process 
		heat for chemical industry
		\item Fuel diversity ($^{235}$U, $^{233}$U, Thorium, U/Pu)
		\item Strong negative temperature feedback of liquid fuel
		\item Passive safety $\Rightarrow$ fuel drains into tanks 
		in emergency
		\item High fuel utilization $\Rightarrow$ less nuclear 
		waste generated
		\item<2> \textbf{On-line (continuous) fuel reprocessing and refueling}
	\end{enumerate}
\end{block}

\end{frame}

\begin{frame}
\frametitle{On-line fuel processing and online refueling pros and cons}
\begin{block}{Advantages}
	\begin{enumerate}
		\item Improved neutrons economy
		\item Better fuel utilization
		\item Smaller excess reactivity
		\item Ability to maintain criticality without shutdown
	\end{enumerate}
\end{block}

\begin{block}{Disadvantages}
	\begin{enumerate}
		\item Chemical separation is challenging
		\item Fuel salt balance in a primary loop complicates operation
		\item Hard to model using most contemporary burnup software
	\end{enumerate}
\end{block}

\end{frame}

\begin{frame}
\frametitle{Continuous vs batch-wise fuel processing  and refueling (1/3)}
		\vspace{-7mm}
\begin{block}{Classical Bateman equation}
	%Core material is circulated to or from the core at specific intervals:
	\begin{align}
	\frac{dN_i}{dt} &= \sum_{m=1}^{M}l_{im}\lambda_mN_m + 
	\phi\sum_{m=1}^{M}f_{im}\sigma_mN_m - (\lambda_i + \phi\sigma_i)N_i + F_i\Big|{i\in [1,M]} \nonumber\\
		N_i &= \mbox{atom density of nuclide i} \nonumber \\
	M &= \mbox{number of nuclides} \nonumber \\
	l_{im} &= \mbox{fraction of decays of nuclide m that result in formation of 
		nuclide i}\nonumber \\
	\lambda_i &= \mbox{radioactive decay constant of nuclide i} \nonumber \\
	\phi &= \mbox{neutron flux, averaged over position and energy} \nonumber \\
	f_{im} &= \mbox{fraction of neutron absorption by nuclide m leading to the 
		formation of nuclide i} \nonumber \\
	\sigma_m &= \mbox{average neutron absorption cross section of nuclide m} 
	\nonumber \\
	F_i &= \mbox{production rate of nuclide i directly from fission}\nonumber
	\end{align}
\end{block}
\end{frame}

\begin{frame}
\frametitle{Continuous vs batch-wise fuel processing  and refueling (2/3)}
\begin{block}{Bateman equation with continuous removals and feed}
	\hspace{-0.4in}
	\begin{align}
	\frac{dN_i}{dt} &= \sum_{m=1}^{M}l_{im}\lambda_mN_m + 
	\phi\sum_{m=1}^{M}f_{im}\sigma_mN_m - (\lambda_i + \phi\sigma_i + \textcolor{red}{r_i - 
	f_i})N_i + F_i\Big|{i\in [1,M]} \nonumber \\
	N_i &= \mbox{atom density of nuclide i} \nonumber \\
	M &= \mbox{number of nuclides} \nonumber \\
	l_{im} &= \mbox{fraction of decays of nuclide m that result in formation of 
		nuclide i}\nonumber \\
	\lambda_i &= \mbox{radioactive decay constant of nuclide i} \nonumber \\
	\phi &= \mbox{neutron flux, averaged over position and energy} \nonumber \\
	f_{im} &= \mbox{fraction of neutron absorption by nuclide m leading to the 
		formation of nuclide i} \nonumber \\
	\sigma_m &= \mbox{average neutron absorption cross section of nuclide m} 
	\nonumber \\
	\textcolor{red}{r_i} & \mkern4mu \textcolor{red}{=} \mbox{\space \textcolor{red}{continuous removal rate of nuclide i from the system}} \nonumber \\
	\textcolor{red}{f_i} & \mkern4mu  \textcolor{red}{=} \mbox{\space \textcolor{red}{continuous feed rate of nuclide i}} \nonumber \\
	F_i &= \mbox{production rate of nuclide i directly from fission}\nonumber
	\end{align}
\end{block}

\end{frame}

\begin{frame}
\frametitle{Continuous vs batch-wise fuel processing and refueling (3/3)}
	\begin{block}{Batch-wise approach}
		Core material is circulated to or from the core at specific intervals		
	\end{block}
           \begin{figure}[t]
	\includegraphics[height=0.45\textwidth]{./images/batch-wise.png}
	\caption{Typical flowchart for batch-wise reprocessing and refueling (figure reproduced from Li et al.\cite{li_optimization_2018}).}
			\end{figure}               
\end{frame}


\begin{frame}
  \frametitle{Fuel salt burnup and reprocessing}
                  \vspace*{-0.05in}
               \begin{enumerate}
               	\item Gaseous fission products (Kr, Xe) escape from the fuel salt
               	\item Noble metals (Mo, Tc) plate out on surfaces
               	\item Rare earth metals should be removed by chemical processing
                \item Most of burnup codes cannot treat fuel movement
               \end{enumerate}

           \begin{figure}[t]
			\hspace*{-0.2in}
                \includegraphics[height=0.48\textwidth]{../figures/periodic_map.png}
		\vspace*{-0.05in}
		\caption{Processing options for MSR fuels (figure reproduced from Ahmed et al.\cite{ahmad_neutronics_2015}).}
     	 \end{figure}               
\end{frame}

\begin{frame}
  \frametitle{Research objectives}
                  \vspace*{-0.1in}

              \begin{block}{Multiphysics simulation of \gls{MSR} (Moltres/MOOSE)\cite{lindsay_introduction_2018}}
               \begin{enumerate}
                \item Demonstrate steady-state and transient coupling of neutron fluxes, precursor drift, and thermal-hydraulics
                \item Implement advective movement of delayed neutron precursors
                \item Demonstrate capabilities with 2D axisymmetric and 3D mesh
                \item Simple transients: change of flow and moderator movement
               \end{enumerate}
               \end{block}


              
\end{frame}
